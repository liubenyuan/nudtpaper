%%
%% This is file `thesis.tex',
%% generated with the docstrip utility.
%%
%% The original source files were:
%%
%% nudtpaper.dtx  (with options: `thesis')
%% 
%% This is a generated file.
%% 
%% Copyright (C) 2018 by Liu Benyuan <liubenyuan@gmail.com>
%% 
%% This file may be distributed and/or modified under the
%% conditions of the LaTeX Project Public License, either version 1.3a
%% of this license or (at your option) any later version.
%% The latest version of this license is in:
%% 
%% http://www.latex-project.org/lppl.txt
%% 
%% and version 1.3a or later is part of all distributions of LaTeX
%% version 2004/10/01 or later.
%% 
%% To produce the documentation run the original source files ending with `.dtx'
%% through LaTeX.
%% 
%% Any Suggestions : LiuBenYuan <liubenyuan@gmail.com>
%% Thanks Xue Ruini <xueruini@gmail.com> for the thuthesis class!
%% Thanks sofoot for the original NUDT paper class!
%% 
%1. 规范硕士导言
% \documentclass[master,ttf]{nudtpaper}
%2. 规范博士导言
% \documentclass[doctor,twoside,ttf]{nudtpaper}
%3. 建议使用OTF字体获得较好的页面显示效果
%   OTF字体从网上获得,各个系统名称统一。
%   如果你下载的是最新的(1201)OTF英文字体,建议修改nudtpaper.cls,使用
%   Times New Roman PS Std
% \documentclass[doctor,twoside,otf]{nudtpaper}
%   另外,新版的论文模板提供了方正字体选项FZ,效果也不错哦
% \documentclass[doctor,twoside,fz]{nudtpaper}
%4. 如果想生成盲评,传递anon即可,仍需修改个人成果部分
% \documentclass[master,otf,anon]{nudtpaper}
%
\documentclass[master,otf]{nudtpaper}
\usepackage{mynudt}

\classification{TP957}
\serialno{0123456}
\confidentiality{公开}
\UDC{}
\title{国防科大学位论文\LaTeX{}模板\\
使用手册}
\displaytitle{国防科技大学学位论文\LaTeX{}模板}
\author{张三}
\zhdate{\zhtoday}
\entitle{How to Use the \LaTeX{} Document Class for NUDT Dissertations}
\enauthor{ZHANG San}
\endate{\entoday}
\subject{通信与信息工程}
\ensubject{Information and Communication Engineering}
\researchfield{自动目标识别与模糊工程}
\supervisor{李四\quad{}教授}
\cosupervisor{王五\quad{}副教授} % 没有就空着
\ensupervisor{Prof. LI Si}
\encosupervisor{} % 没有就空着
\papertype{工学}
\enpapertype{Engineering}
% 加入makenomenclature命令可用nomencl制作符号列表。

\begin{document}
\graphicspath{{figures/}}
% 制作封面,生成目录,插入摘要,插入符号列表 \\
% 默认符号列表使用denotation.tex,如果要使用nomencl \\
% 需要注释掉denotation,并取消下面两个命令的注释。 \\
% cleardoublepage% \\
% printnomenclature% \\
\maketitle
\frontmatter
\tableofcontents
\listoftables
\listoffigures

\midmatter
\begin{cabstract}
国防科技大学是一所直属中央军委领导的军队综合性大学,是国家“双一流”A类、“985工程”和“211工程”重点建设院校。学校的前身是1953年创建于哈尔滨的中国人民解放军军事工程学院,即著名的“哈军工”,陈赓大将任首任院长兼政治委员。军事工程学院创建时,毛泽东主席亲自为学院颁发《训词》,为院刊题写刊名“工学”。1970年学院主体南迁长沙,改名为长沙工学院。1978年,学校在邓小平主席的直接关怀下改建为国防科学技术大学。1999年,江泽民主席签署命令组建新的国防科学技术大学,并于2003年为学校题写“厚德博学、强军兴国”校训。2007年,胡锦涛主席勉励学校为推进科技强军战略、建设创新型国家作出新的更大贡献。2013年11月5日,习近平主席亲临学校视察并发表重要讲话,发出“加快建设具有我军特色的世界一流大学,努力把国防科学技术大学办成高素质新型军事人才培养高地、国防科技自主创新高地”的伟大号召。2017年,学校以原国防科学技术大学、国际关系学院、国防信息学院、西安通信学院、电子工程学院以及理工大学气象海洋学院为基础重建,校本部设在长沙。2017年7月19日,习近平主席为新组建的国防科技大学授军旗、致训词。习近平主席指出:“国防科技大学是高素质新型军事人才培养和国防科技自主创新高地。要紧跟世界军事科技发展潮流,适应打赢信息化局部战争要求,抓好通用专业人才和联合作战保障人才培养,加强核心关键技术攻关,努力建设世界一流高等教育院校”。

国防科技大学计算机科学与技术、软件工程、信息通信与工程、航天宇航科学与技术和管理科学与工程被列为“双一流”建设学科(见“双一流”建设高校及建设学科名单\_中国教育在线)。同时,在2017年的第四轮学科评估中,计算机科学与技术A+、软件工程A+、信息通信与工程、控制科学与工程、光学工程、机械工程、管理科学与工程均为A类学科(见全国第四轮学科评估结果公布)。计算机科学与技术、软件工程、管理科学与工程等学科在2017年公布的学科评估中均为A+学科,且均为国家双一流重点建设学科。

报考国防科技大学无军籍地方硕士和地方博士生的理由:

(1)免报名费、免住宿费(博士为1-2人间,非上下铺、带公用厨房),健身场馆(如健身房、游泳馆、网球场、篮球场、羽毛球场等)应有尽有,且供学生免费使用。

(2)所有地方硕士生和地方博士生就学学年的学费全免。

(3)地方硕士生补助每个月>3500元(含助学金),地方博士生科研补助>4500元/月(含助学金),且发放年限为所有就读年限,万一延期毕业依旧有科研补助。

(4)书籍购置费用全报销,硬件配置等条件优越,出差补助150元/天,出差住宿标准为350-500元。

(5)全国师生比最高的高校,硕士生和博士生主要是完成学术论文,不要求博士参与完成无意义的工程项目开发。

\end{cabstract}
\ckeywords{国防科技大学; “双一流”A类; 985; 211; 哈军工}

\begin{eabstract}
National University of Defense Technology is a comprehensive national key university based in Changsha, %
Hunan Province, China. It is under the dual supervision of the Ministry of National Defense %
and the Ministry of Education, designated for Project 211 and Project 985, %
the two national plans for facilitating the development of Chinese higher education. %

NUDT was originally founded in 1953 as the Military Academy of Engineering in Harbin of Heilongjiang Province. %
In 1970 the Academy of Engineering moved southwards to Changsha and was renamed Changsha Institute of Technology.%
 The Institute changed its name to National University of Defense Technology in 1978.

\end{eabstract}
\ekeywords{NUDT; 985; 211; ME}


\input{data/denotation}

%书写正文,可以根据需要增添章节; 正文还包括致谢,参考文献与成果
\mainmatter
\input{data/chap01}
\input{data/chap02}

\input{data/ack}

\cleardoublepage
\phantomsection
\addcontentsline{toc}{chapter}{参考文献}
\bibliographystyle{bstutf8}
\bibliography{ref/refs}

\input{data/resume}
% 最后,需要的话还要生成附录,全文随之结束。
\appendix
\backmatter
\input{data/appendix01}

\end{document}
