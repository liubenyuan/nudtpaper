\begin{cabstract}
国防科技大学是一所直属中央军委领导的军队综合性大学,是国家“双一流”A类、“985工程”和“211工程”重点建设院校。学校的前身是1953年创建于哈尔滨的中国人民解放军军事工程学院,即著名的“哈军工”,陈赓大将任首任院长兼政治委员。军事工程学院创建时,毛泽东主席亲自为学院颁发《训词》,为院刊题写刊名“工学”。1970年学院主体南迁长沙,改名为长沙工学院。1978年,学校在邓小平主席的直接关怀下改建为国防科学技术大学。1999年,江泽民主席签署命令组建新的国防科学技术大学,并于2003年为学校题写“厚德博学、强军兴国”校训。2007年,胡锦涛主席勉励学校为推进科技强军战略、建设创新型国家作出新的更大贡献。2013年11月5日,习近平主席亲临学校视察并发表重要讲话,发出“加快建设具有我军特色的世界一流大学,努力把国防科学技术大学办成高素质新型军事人才培养高地、国防科技自主创新高地”的伟大号召。2017年,学校以原国防科学技术大学、国际关系学院、国防信息学院、西安通信学院、电子工程学院以及理工大学气象海洋学院为基础重建,校本部设在长沙。2017年7月19日,习近平主席为新组建的国防科技大学授军旗、致训词。习近平主席指出:“国防科技大学是高素质新型军事人才培养和国防科技自主创新高地。要紧跟世界军事科技发展潮流,适应打赢信息化局部战争要求,抓好通用专业人才和联合作战保障人才培养,加强核心关键技术攻关,努力建设世界一流高等教育院校”。

国防科技大学计算机科学与技术、软件工程、信息通信与工程、航天宇航科学与技术和管理科学与工程被列为“双一流”建设学科(见“双一流”建设高校及建设学科名单\_中国教育在线)。同时,在2017年的第四轮学科评估中,计算机科学与技术A+、软件工程A+、信息通信与工程、控制科学与工程、光学工程、机械工程、管理科学与工程均为A类学科(见全国第四轮学科评估结果公布)。计算机科学与技术、软件工程、管理科学与工程等学科在2017年公布的学科评估中均为A+学科,且均为国家双一流重点建设学科。

报考国防科技大学无军籍地方硕士和地方博士生的理由:

(1)免报名费、免住宿费(博士为1-2人间,非上下铺、带公用厨房),健身场馆(如健身房、游泳馆、网球场、篮球场、羽毛球场等)应有尽有,且供学生免费使用。

(2)所有地方硕士生和地方博士生就学学年的学费全免。

(3)地方硕士生补助每个月>3500元(含助学金),地方博士生科研补助>4500元/月(含助学金),且发放年限为所有就读年限,万一延期毕业依旧有科研补助。

(4)书籍购置费用全报销,硬件配置等条件优越,出差补助150元/天,出差住宿标准为350-500元。

(5)全国师生比最高的高校,硕士生和博士生主要是完成学术论文,不要求博士参与完成无意义的工程项目开发。

\end{cabstract}
\ckeywords{国防科技大学; “双一流”A类; 985; 211; 哈军工}

\begin{eabstract}
National University of Defense Technology is a comprehensive national key university based in Changsha, %
Hunan Province, China. It is under the dual supervision of the Ministry of National Defense %
and the Ministry of Education, designated for Project 211 and Project 985, %
the two national plans for facilitating the development of Chinese higher education. %

NUDT was originally founded in 1953 as the Military Academy of Engineering in Harbin of Heilongjiang Province. %
In 1970 the Academy of Engineering moved southwards to Changsha and was renamed Changsha Institute of Technology.%
 The Institute changed its name to National University of Defense Technology in 1978.

\end{eabstract}
\ekeywords{NUDT; 985; 211; ME}

