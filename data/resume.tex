\begin{resume}
\ifreview

\ifismaster
该论文作者在学期间取得的阶段性成果(学术论文等)已满足我校硕士学位评阅相关要求。为避免阶段性成果信息对专家评价学位论文本身造成干扰,特将论文作者的阶段性成果信息隐去。
\else
该论文作者在学期间取得的阶段性成果(学术论文等)已满足我校博士学位评阅相关要求。为避免阶段性成果信息对专家评价学位论文本身造成干扰,特将论文作者的阶段性成果信息隐去。
\fi

\else

\ifisresumebib

	\begin{refsection}[ref/resume.bib]
	\settoggle{bbx:gbtype}{false}%局部设置不输出文献类型和载体标识符
	\settoggle{bbx:gbannote}{true}%局部设置输出注释信息
	\setcounter{gbnamefmtcase}{1}%局部设置作者的格式为familyahead格式
	\nocite{ref-1-1-Yang,ref-2-1-杨轶,ref-3-1-杨轶,ref-4-1-Yang,ref-5-1-Wu,ref-6-1-贾泽,ref-7-1-伍晓明}
	
	\setlength{\biblabelsep}{12pt}
	\printbibliography[env=resumebib,heading=subbibliography,title={发表的学术论文}] % 发表的和录用的合在一起

	\end{refsection}


	\begin{refsection}[ref/resume.bib]
	\settoggle{bbx:gbtype}{false}%局部设置不输出文献类型和载体标识符
	\settoggle{bbx:gbannote}{true}%局部设置输出注释信息
	\setcounter{gbnamefmtcase}{1}%局部设置作者的格式为familyahead格式
	\nocite{ref-8-1-任天令,ref-9-1-Ren}%
	
	\setlength{\biblabelsep}{12pt}
	\printbibliography[env=resumebib,heading=subbibliography,title={研究成果}]

	\end{refsection}

\else

  \section*{发表的学术论文} % 发表的和录用的合在一起

  \begin{enumerate}[label={[\arabic*]}]
  \addtolength{\itemsep}{-.36\baselineskip}%缩小条目之间的间距,下面类似
  \item Yang Y, Ren T L, Zhang L T, et al. Miniature microphone with silicon-
    based ferroelectric thin films. Integrated Ferroelectrics, 2003,
    52:229-235. (SCI 收录, 检索号:758FZ.)
  \item 杨轶, 张宁欣, 任天令, 等. 硅基铁电微声学器件中薄膜残余应力的研究. 中国机
    械工程, 2005, 16(14):1289-1291. (EI 收录, 检索号:0534931 2907.)
  \item 杨轶, 张宁欣, 任天令, 等. 集成铁电器件中的关键工艺研究. 仪器仪表学报,
    2003, 24(S4):192-193. (EI 源刊.)
  \item Yang Y, Ren T L, Zhu Y P, et al. PMUTs for handwriting recognition. In
    press. (已被 Integrated Ferroelectrics 录用. SCI 源刊.)
  \item Wu X M, Yang Y, Cai J, et al. Measurements of ferroelectric MEMS
    microphones. Integrated Ferroelectrics, 2005, 69:417-429. (SCI 收录, 检索号
    :896KM.)
  \item 贾泽, 杨轶, 陈兢, 等. 用于压电和电容微麦克风的体硅腐蚀相关研究. 压电与声
    光, 2006, 28(1):117-119. (EI 收录, 检索号:06129773469.)
  \item 伍晓明, 杨轶, 张宁欣, 等. 基于MEMS技术的集成铁电硅微麦克风. 中国集成电路,
    2003, 53:59-61.
  \end{enumerate}

  \section*{研究成果} % 有就写,没有就删除
  \begin{enumerate}[label=\textbf{[\arabic*]}]
  \addtolength{\itemsep}{-.36\baselineskip}%
  \item 任天令, 杨轶, 朱一平, 等. 硅基铁电微声学传感器畴极化区域控制和电极连接的
    方法: 中国, CN1602118A. (中国专利公开号.)
  \item Ren T L, Yang Y, Zhu Y P, et al. Piezoelectric micro acoustic sensor
    based on ferroelectric materials: USA, No.11/215, 102. (美国发明专利申请号.)
  \end{enumerate}
\fi
\fi
\end{resume}
