% \iffalse meta-comment
%
% Copyright (C) \the\year by Liu Benyuan <liubenyuan@gmail.com>
% This file may be distributed and/or modified under the
% conditions of the LaTeX Project Public License, either
% version 1.2 of this license or (at your option) any later
% version. The latest version of this license is in:
%
% http://www.latex-project.org/lppl.txt
%
% and version 1.2 or later is part of all distributions of
% LaTeX version 1999/12/01 or later.
%
% \fi
%
% \iffalse
% <package>\NeedsTeXFormat{LaTeX2e}[1999/12/01]
% <package>\ProvidesPackage{nudtpaper}
% <package>[2017/06/15 v2.7 By Liu Benyuan <liubenyuan@gmail.com>]
%<*driver>
\ProvidesFile{nudtpaper.dtx}[2017/06/15 v2.7 NUDT]
\documentclass[11pt]{ltxdoc}
\usepackage{nudtx}
\EnableCrossrefs
\CodelineIndex
\RecordChanges
\begin{document}
  \DocInput{\jobname.dtx}
\end{document}
%</driver>
% \fi
%
% \def\thuthesis{\textsc{Thu}\-\textsc{Thesis}}
% \def\nudtpaper{\textsc{Nudt}\-\textsc{Paper}}
%
% \CheckSum{1245}
% \CharacterTable
%  {Upper-case    \A\B\C\D\E\F\G\H\I\J\K\L\M\N\O\P\Q\R\S\T\U\V\W\X\Y\Z
%   Lower-case    \a\b\c\d\e\f\g\h\i\j\k\l\m\n\o\p\q\r\s\t\u\v\w\x\y\z
%   Digits        \0\1\2\3\4\5\6\7\8\9
%   Exclamation   \!     Double quote  \"     Hash (number) \#
%   Dollar        \$     Percent       \%     Ampersand     \&
%   Acute accent  \'     Left paren    \(     Right paren   \)
%   Asterisk      \*     Plus          \+     Comma         \,
%   Minus         \-     Point         \.     Solidus       \/
%   Colon         \:     Semicolon     \;     Less than     \<
%   Equals        \=     Greater than  \>     Question mark \?
%   Commercial at \@     Left bracket  \[     Backslash     \\
%   Right bracket \]     Circumflex    \^     Underscore    \_
%   Grave accent  \`     Left brace    \{     Vertical bar  \|
%   Right brace   \}     Tilde         \~}
%
% \changes{v0.99}{2009/08/12}{Initial Release}
%
% \GetFileInfo{\jobname.dtx}
%
% \DoNotIndex{\begin,\end,\begingroup,\endgroup}
% \DoNotIndex{\ifx,\ifdim,\ifnum,\ifcase,\else,\or,\fi}
% \DoNotIndex{\let,\def,\xdef,\newcommand,\renewcommand}
% \DoNotIndex{\expandafter,\csname,\endcsname,\relax,\protect}
% \DoNotIndex{\Huge,\huge,\LARGE,\Large,\large,\normalsize}
% \DoNotIndex{\small,\footnotesize,\scriptsize,\tiny}
% \DoNotIndex{\normalfont,\bfseries,\slshape,\interlinepenalty}
% \DoNotIndex{\hfil,\par,\hskip,\vskip,\vspace,\quad}
% \DoNotIndex{\centering,\raggedright}
% \DoNotIndex{\c@secnumdepth,\@startsection,\@setfontsize}
% \DoNotIndex{\ ,\@plus,\@minus,\p@,\z@,\@m,\@M,\@ne,\m@ne}
% \DoNotIndex{\@@par,\DeclareOperation,\RequirePackage,\LoadClass}
% \DoNotIndex{\AtBeginDocument,\AtEndDocument}
%
% \IndexPrologue{\section*{索引}%
%    \addcontentsline{toc}{section}{索~~~~引}}
% \GlossaryPrologue{\section*{修改记录}%
%    \addcontentsline{toc}{section}{修改记录}}
%
% \renewcommand{\abstractname}{摘~~要}
% \renewcommand{\contentsname}{目~~录}
%
% \title{\textsc{NUDTpaper:}\,\,NUDT研究生学位论文\LaTeX{}模板使用手册\thanks{NUDT \LaTeX{} Thesis Template}}
% \author{刘本源 \\ \texttt{Liubenyuan@gmail.com}}
% \date{\fileversion\ (\filedate)}
%
% \maketitle
% \thispagestyle{empty}
%
% \begin{abstract}
% 本模板旨在提供规范的国防科技大学\LaTeX{}写作模板环境,
% 现支持硕士/博士学位论文格式,可以自动生成盲评、制作A3封面。
% \end{abstract}
%
% \vspace{2cm}
% \def\abstractname{免责声明}
% \begin{abstract}\noindent
% \begin{enumerate}
% \item 本模板的发布遵守 \LaTeX{} Project Public License,使用前请认真阅读协议内容
% \item 本模板创立参照官方严格的论文写作手册,并同时参照硕士/博士学位论文\textbf{doc}文档对比修改
% \item 国防科技大学对论文写作提供写作指南与官方\textbf{doc}模板,
% 同时提供官方的\LaTeX{}模板,本模板的出发点是方便大家使用专业的高效的论文书写工具,
% 其有点在于注重排版质量、命令规范、使用方便、更新及时,符合论文撰写说明。
% 但任何由于使用本模板而引起的论文格式审查问题均与本模板作者无关。
% \item 任何个人或组织均可以本模板为基础进行修改、扩展,生成新的专用模板,但请严格遵
% 守\LaTeX{} Project Public License 协议
% \item 欢迎提出修改意见
% \end{enumerate}
% \end{abstract}
%
% \clearpage
% \tableofcontents
%
% \clearpage
% \pagenumbering{arabic}
% \pagestyle{mainpage}
%
% \section{快速上手}
% \begin{description}
% \item[安装\TeX] 下载最新的\TeX{}live或者C\TeX{}并安装
% \item[字体] 用户需要具备\verb|simsun.ttf|, \verb|simhei.ttf|, \verb|simkai.ttf|,
% \verb|STZHONGS.TTF|, 上述字体都是windows自带的; 除此之外,在网上搜索(或者C\TeX{}
% 论坛)``Adobe Opentype 中文字体'',一搜一大把,确保下载下来Adobe的四款OTF字体:
% 宋,黑,仿宋,楷体。Linux用户可将上述字体复制到\verb|/usr/share/fonts/TTF|下。
% 最新更新:用户可以考虑使用方正字体生成更为漂亮(颜色深均匀)的论文排版。
% 这三个选项分别为\verb|ttf|、\verb|otf|和\verb|fz|。
% \item[试一试] 解压缩下载的模板,双击makepdf.bat(祈祷一下),如果生成了
% \verb|thesis.pdf|$\rightarrow$
% \item[那么我的那些常用的包都在么?] 你会想我的{\bf Trans}论文可以无缝
% 的复制过来么? 对于这一点,你可以修改\verb|mynudt.sty|来实现。但是{\hei 注意},大部分包
% 都在模板中了,而且{\hei 切记切记},不要擅自改动字体等版面设计,我们继续看$\rightarrow$
% \item[咦,数学公式不是很美观呀] 笔者{\hei 强烈}建议用户使用{\bf mtpro2}宏包的,怎么使用,
% 又有哪些好处,参见bookzh.sty吧!不会错的。好了,我们专注于内容本身吧$\rightarrow$
% \item[开始写了] 所有文件均采用UTF8编码,因此要保证你的\TeX{}编辑器
% (winedt, texworks, texmaker, vim, 记事本($\cdots{}$)等)支持这种编码,
% (经过一番搜索设置后)打开\verb|thesis.tex|,如果看到的是中文$\rightarrow$
% \item[漫长的写作] 手边准备着\LaTeX{}的常用帮助文档(数学,图表,引用等),
% 结合你喜欢的文献管理软件(JabRef等), 漫长的\texttt{编辑,编译,修改,编辑,
% 编译$\cdots$}过程之后,突然有一天发现你写完了$\rightarrow$
% \item[校订] 经过老师师兄师弟师妹齐心协力校正之后,你所做的只是:
% \texttt{生成明评论文,制作明评封面,生成盲评论文,制作盲评封面},
% 装订,上交$\rightarrow$
% \end{description}
% {\color{magenta} Done!}
%
% \section{模板介绍}
%
% \textsc{NUDTpaper} 旨在帮助并且推广\LaTeX{}在国防科技大学论文中的应用,
% 本文将尽可能帮助用户掌握\textsc{NUDTpaper}的安装方法,
% 如果仍旧有不清晰的地方可以参考样例文件或者
% 给作者邮件\footnote{liubenyuan@gmail.com},感兴趣的同学可以帮忙维护模板,
% 这个模板首先符合官方的设计要求,希望同学们在使用后能够提出你们的修改意见。
% 该模板很大程度上参考了6院黄老师sofoot的国防科大博士论文模板,
% 哈工大的\LaTeX{}模板以及清华的Thuthesis
% \footnote{主页:\url{http://thuthesis.sourceforge.net}},
% 有很多使用的帮助、\verb|.cls|中的命令以及版面设置均来自Thuthesis和sofoot的模板,
% 对此的引用表示感谢。
%
% {\color{blue}\fs 模板的作用在于减轻论文写作过程中格式调整的时间,
% 其前提就是遵守模板的用法,不提倡手动更改格式,不建议正文中使用
% 手动调节版面的命令,尤其禁止修改行距和使用\verb|\normalsize|,
% 否则即使使用了\textsc{NUDTpaper}也难以保证输出的论文符合学校规范。}
%
% \section{安装}
% \label{sec:install}
%
% \subsection{下载}
% \textsc{NUDTpaper} 主页:\url{http://nudtpaper.googlecode.com}。
% 模板的更新信息发布在\href{http://bbs.ctex.org}{Ctex论坛}。
% \nudtpaper{}的开发版本同样可以在\textsc{gitorious}上获得。
%
% \subsection{模板的组成部分}
% 下表列出了 \nudtpaper{} 的主要文件及其功能介绍,学习模板的最好办法
% 就是参考thesis.pdf!
% \begin{center}
% \begin{longtable}{l|p{8cm}}
% \toprule
% {\hei 文件(夹)} & {\hei 功能描述}\\\midrule
% \endfirsthead
% \toprule
% {\hei 文件(夹)} & {\hei 功能描述}\\\midrule
% \endhead
% \endfoot
% \endlastfoot
% nudtpaper.ins & 模板驱动文件 \\
% nudtpaper.dtx & 模板文档代码的混合文件\\
% nudtpaper.cls & 模板类文件\\
% nudtpaper.cfg & 模板配置文件\\
% thesis.bib & 参考文献样式文件\\
% \hline
% mynudt.sty & 在这里添加你自己的宏包 \\
% thesis.tex & 示例文档主文件\\
% ref/ & 示例文档参考文献目录\\
% data/ & 示例文档章节具体内容\\
% figures/ & 示例文档图片路径\\
% \textbf{nudtpaper.pdf} & 用户手册(本文档)\\
% \textbf{thesis.pdf} & 示例文档 \\
% \bottomrule
% \end{longtable}
% \end{center}
%
% \subsection{\TeX{}系统的选择}
% 有网络环境的用户推荐安装\href{http://www.tug.org/texlive}{\TeX{}live},
% \href{http://miktex.org}{MiKTeX}或者\href{http://www.ctex.org}{C\TeX},
% 对于无网络环境的,主要是针对教研室用户,推荐{\TeX{}live}或者C\TeX{}完整版,安装
% 过程很简单,一路下一步即可,但是需要\textbf{注意:}
%
% \begin{description}
% \item[字体] TTF选项默认调用Windows系统字体,其中楷体、仿宋需要安装Office;OTF选项需要
% Adobe的商业字体(可以使你的论文更加漂亮!),这些中文字体(宋,黑,仿宋,楷体)可以从
% \href{http://dl.getdropbox.com/u/857066/adobe_chinese_otf.7z}{这里下载},
% 如果上述链接不能使用,请搜索\textsc{Adobe Opentype 中文字体}自行下载。
% 英文字体使用Windows自带。起始更推荐几款Times(Arial)类似的OTF英文字体,可以使用
% 更多排版、段落的字体特性。
% \item[粗宋] 模板中在需要宋体加黑的地方需要使用\textbf{华文中宋}, 即STZHONGS.TTF。
% \item[xeCJK] 无网络环境中,C\TeX{}完整版和\TeX{}live最新版都包括了需要的xeCJK版本。
% \end{description}
%
% \subsection{使用模板}
% \label{sec:install-cls}
% {\hei 注:默认的发行版本已经包含了可以使用的模板环境,
% 包括编译好的cls以及论文样例源文件,
% 想快速上手的话,可以直接参看\verb|thesis.tex|,进行修改。
% 写作的过程就是将你的论文的内容放到\verb|data|文件夹中,
% 图片放到\verb|figures|文件夹中,用\textsc{jabref}修改\verb|thesis.bib|即可。}
%
% 当用户需要编译生成自己的PDF版论文时,需要依次输入:(注意了,如果不是使用nomencl,
% 则无需使用第二个命令)
% \begin{shell}
% $ xelatex thesis
% $ # makeindex -s nomencl.ist -o thesis.nls thesis.nlo
% $ bibtex thesis
% $ bibtex thesis
% $ xelatex thesis
% $ xelatex thesis
% \end{shell}
%
% 而为了简化用户使用,模板中提供了快捷脚本文件:
% \begin{shell}
% # 下面命令可以直接生成thesis.pdf,你可能只需要这步
% C:\> makepdf.bat
% # linux用户可以直接使用makefile
% $ make pdf
% \end{shell}
% 现在,就要进入激动人心的写作过程了。
%
% \section{使用说明}
% \label{sec:how-to-use}
% 首先,一篇论文(电子信息工程专业为例),主要的构成就是
% 封面导言,正文,表格,图片,公式,
% 交叉引用及文献索引这五部分,下面将分别详细讲解。
%
% \label{sec:howtoask}
% 在开始之前,先问自己几个问题:
% \begin{compactenum}
% \item 我是不是已经掌握了 \LaTeX{} 基础知识?
% \item 我是不是认真地阅读了模板文档?
% \item 周围有没有同学可以帮我?
% \end{compactenum}
% 更推荐用户去阅读示例文档的源代码,改写会给你一个快速的开始。
%
% \subsection{示例文件}
% \label{sec:example}
% 该示例文件是顶层的文件,包括论文属性设置、章节的安排、参考文献附录等。
% 细节用户可以参考\verb|thesis.tex|和\verb|data/|文件夹。
%
% \subsubsection{模板选项}
%
% 论文的第一句话是调用模板:
% \changes{v2.0}{2010/11/10}{增加盲评的说明}
%
%    \begin{macrocode}
%<thesis>%1. 规范硕士导言
%<thesis>% \documentclass[master,ttf]{nudtpaper}
%<thesis>%2. 规范博士导言
%<thesis>% \documentclass[doctor,twoside,ttf]{nudtpaper}
%<thesis>%3. 建议使用OTF字体获得较好的页面显示效果
%<thesis>%   OTF字体从网上获得,各个系统名称统一。
%<thesis>%   如果你下载的是最新的(1201)OTF英文字体,建议修改nudtpaper.cls,使用
%<thesis>%   Times New Roman PS Std
%<thesis>% \documentclass[doctor,twoside,otf]{nudtpaper}
%<thesis>%   另外,新版的论文模板提供了方正字体选项FZ,效果也不错哦
%<thesis>% \documentclass[doctor,twoside,fz]{nudtpaper}
%<thesis>%4. 如果想生成盲评,传递anon即可,仍需修改个人成果部分
%<thesis>% \documentclass[master,otf,anon]{nudtpaper}
%<thesis>%
%    \end{macrocode}
%    \begin{macrocode}
%<*thesis>
\documentclass[master,otf]{nudtpaper}
\usepackage{mynudt}

%</thesis>
%    \end{macrocode}
%
% 模板的参数设置(开关)描述见:
%
%\begin{description}
%\item[master,doctor]
% 硕士论文用master,博士论文就用doctor
%\item[twoside]
% 指定论文为单面打印还是双面打印,当使用\verb|twoside|选项之后,
% 论文会将章节开在奇数页右手边,默认为\verb|openany|单面打印。
%\item[ttf,otf]
% 决定使用何种字体,TTF默认使用Windows自带的字体,而OTF则使用Adobe的字体(需要下载),
% TTF字体的优势是满足学校论文对于字体的要求,缺点是制作出来的PDF文件在浏览时可能发虚,
% 而OTF字体屏幕显示饱满,而且字体有很多选项可以方便\XeTeX{}排版。推荐使用\textbf{otf}
% 选项。不论何种选项,都需要安装宋体中宋(STZHONGSONG)字体(Windows自带)。
%\item[anon]
% 是否为盲评版本,如需盲评,请加上anon。
%\end{description}
%
% 如果需要使用自己定义的命令、宏包,请放于\verb|mynudt.sty|中。
% 事实上,该文件中已经添加了很多有用的宏包和命令,你可以参照修改。
% 这些之所以没有放到模板中,一则为了简洁,二则赋予用户在格式之外更多的自由。
% 里面的宏包有:代码高亮、算法环境、向量命令等,请仔细查看。
%
% 样例文件默认的是硕士论文(master),OTF字体(otf)。
%
% \subsubsection{封面导言}
% 官方模板中设计论文题目、作者等信息可以跟填空一样完成:
%
% \begin{description}
% \item[论文封头]
% 主要有四部分内容,中图分类号,学号,论文密级和UDC。
% 密级分为:\textbf{秘密} 或者 \textbf{公开}。
%    \begin{macrocode}
%<*thesis>
\classification{TP957}
\serialno{0123456}
\confidentiality{公开}
\UDC{}
%</thesis>
%    \end{macrocode}
%
% \item[论文题目,作者,日期]
% 分别包括中文和英文两部分,由于论文题目可能超过1行,
% 我们提供额外的一个命令\verb|\displaytitle|用来在
% 授权书中填入(限定为)单行的题目; 中文日期需要中文输入大写,英文日期为月年,
% 在论文最终完成后,请\textbf{手动}设定日期。
% \changes{v2.5}{2015/05/11}{修改英文作者和导师的格式,姓大写}
% \changes{v2.5}{2015/05/11}{英文导师前缀为Prof.}
%
%    \begin{macrocode}
%<*thesis>
\title{国防科大学位论文\LaTeX{}模板\\
使用手册}
\displaytitle{国防科技大学学位论文\LaTeX{}模板}
\author{张三}
\zhdate{\zhtoday}
\entitle{How to Use the \LaTeX{} Document Class for NUDT Dissertations}
\enauthor{ZHANG San}
\endate{\entoday}
%</thesis>
%    \end{macrocode}
%
% \item[论文分类及其他]
% 主要是作者的学科类别,研究方向,导师信息等。
% 每一项都包括中英文信息:
%
%    \begin{macrocode}
%<*thesis>
\subject{通信与信息工程}
\ensubject{Information and Communication Engineering}
\researchfield{自动目标识别与模糊工程}
\supervisor{李四\quad{}教授}
\cosupervisor{王五\quad{}副教授} % 没有就空着
\ensupervisor{Prof. LI Si}
\encosupervisor{} % 没有就空着
\papertype{工学}
\enpapertype{Engineering}
%</thesis>
%    \end{macrocode}
%
% \item[中英文摘要]
% 论文中需要写中文以及英文摘要,页码为小写罗马字母,关键字为黑体,
% 英文关键字为\verb|Arial|,
% 模板中定义了相关环境\verb|\cabstract|以及\verb|\eabstract|来书写摘要,
% 以及\verb|\ckeywords|以及\verb|\ekeywords|来写关键字。
% 建议用户将摘要单独放在在\verb|abstract.tex|文件中,
% 在正文中\verb|\begin{cabstract}
国防科技大学是一所直属中央军委领导的军队综合性大学,是国家“双一流”A类、“985工程”和“211工程”重点建设院校。学校的前身是1953年创建于哈尔滨的中国人民解放军军事工程学院,即著名的“哈军工”,陈赓大将任首任院长兼政治委员。军事工程学院创建时,毛泽东主席亲自为学院颁发《训词》,为院刊题写刊名“工学”。1970年学院主体南迁长沙,改名为长沙工学院。1978年,学校在邓小平主席的直接关怀下改建为国防科学技术大学。1999年,江泽民主席签署命令组建新的国防科学技术大学,并于2003年为学校题写“厚德博学、强军兴国”校训。2007年,胡锦涛主席勉励学校为推进科技强军战略、建设创新型国家作出新的更大贡献。2013年11月5日,习近平主席亲临学校视察并发表重要讲话,发出“加快建设具有我军特色的世界一流大学,努力把国防科学技术大学办成高素质新型军事人才培养高地、国防科技自主创新高地”的伟大号召。2017年,学校以原国防科学技术大学、国际关系学院、国防信息学院、西安通信学院、电子工程学院以及理工大学气象海洋学院为基础重建,校本部设在长沙。2017年7月19日,习近平主席为新组建的国防科技大学授军旗、致训词。习近平主席指出:“国防科技大学是高素质新型军事人才培养和国防科技自主创新高地。要紧跟世界军事科技发展潮流,适应打赢信息化局部战争要求,抓好通用专业人才和联合作战保障人才培养,加强核心关键技术攻关,努力建设世界一流高等教育院校”。

国防科技大学计算机科学与技术、软件工程、信息通信与工程、航天宇航科学与技术和管理科学与工程被列为“双一流”建设学科(见“双一流”建设高校及建设学科名单\_中国教育在线)。同时,在2017年的第四轮学科评估中,计算机科学与技术A+、软件工程A+、信息通信与工程、控制科学与工程、光学工程、机械工程、管理科学与工程均为A类学科(见全国第四轮学科评估结果公布)。计算机科学与技术、软件工程、管理科学与工程等学科在2017年公布的学科评估中均为A+学科,且均为国家双一流重点建设学科。

报考国防科技大学无军籍地方硕士和地方博士生的理由:

(1)免报名费、免住宿费(博士为1-2人间,非上下铺、带公用厨房),健身场馆(如健身房、游泳馆、网球场、篮球场、羽毛球场等)应有尽有,且供学生免费使用。

(2)所有地方硕士生和地方博士生就学学年的学费全免。

(3)地方硕士生补助每个月>3500元(含助学金),地方博士生科研补助>4500元/月(含助学金),且发放年限为所有就读年限,万一延期毕业依旧有科研补助。

(4)书籍购置费用全报销,硬件配置等条件优越,出差补助150元/天,出差住宿标准为350-500元。

(5)全国师生比最高的高校,硕士生和博士生主要是完成学术论文,不要求博士参与完成无意义的工程项目开发。

\end{cabstract}
\ckeywords{国防科技大学; “双一流”A类; 985; 211; 哈军工}

\begin{eabstract}
National University of Defense Technology is a comprehensive national key university based in Changsha, %
Hunan Province, China. It is under the dual supervision of the Ministry of National Defense %
and the Ministry of Education, designated for Project 211 and Project 985, %
the two national plans for facilitating the development of Chinese higher education. %

NUDT was originally founded in 1953 as the Military Academy of Engineering in Harbin of Heilongjiang Province. %
In 1970 the Academy of Engineering moved southwards to Changsha and was renamed Changsha Institute of Technology.%
 The Institute changed its name to National University of Defense Technology in 1978.

\end{eabstract}
\ekeywords{NUDT; 985; 211; ME}

|即可。其格式为:
%
% \begin{example}
% \begin{cabstract}
% 中文摘要
% \end{cabstract}
% \ckeywords{关键字}
%
% \begin{eabstract}
% Abstract
% \end{eabstract}
% \ekeywords{Key}
% \end{example}
% \end{description}
%
% \subsubsection{框架构成}
%
% 在定义完论文元素之后,就可以开始写论文正文了。用\LaTeX{}写论文的文件目录构成
% 可以很随意,模板中将图形文件单独放到一个目录中\verb|figure|中,论文正文各个
% 章节置于\verb|data|中;当然也以以\verb|chapter|为目录。
%\changes{v2.2}{2011/05/27}{使用nomencl包管理符号列表}
%\changes{v2.2}{2011/07/08}{默认使用nomencl管理参考文献}
%\changes{v2.2}{2011/09/10}{回复原先使用的denotation方式添加符号列表}
%
% 如果使用nomencl制作符号列表,在文档开始前要加入\verb|\makenomenclature|命令
% 默认还是使用denotation的方式。nomencl可以参考第二章相关章节,而denote方式
% 请参考\verb|data/denotation.tex|文件(简单的列表环境)。
%
%<thesis>% 加入makenomenclature命令可用nomencl制作符号列表。
%
%    \begin{macrocode}
%<*thesis>

\begin{document}
\graphicspath{{figures/}}
%</thesis>
%    \end{macrocode}
% 制作完封面后就是正文四大部分了,分别为:
%
% \begin{compactenum}
% \item frontmatter: 生成目录,图目录,表目录
% \item midmatter: 摘要,符号列表
% \item mainmatter: 正文,致谢,文献,成果
% \item backmatter: 附录
% \end{compactenum}
%
%<thesis>% 制作封面,生成目录,插入摘要,插入符号列表 \\
%<thesis>% 默认符号列表使用denotation.tex,如果要使用nomencl \\
%<thesis>% 需要注释掉denotation,并取消下面两个命令的注释。 \\
%<thesis>% cleardoublepage% \\
%<thesis>% printnomenclature% \\
%
%    \begin{macrocode}
%<*thesis>
\maketitle
\frontmatter
\tableofcontents
\listoftables
\listoffigures

\midmatter
\begin{cabstract}
国防科技大学是一所直属中央军委领导的军队综合性大学,是国家“双一流”A类、“985工程”和“211工程”重点建设院校。学校的前身是1953年创建于哈尔滨的中国人民解放军军事工程学院,即著名的“哈军工”,陈赓大将任首任院长兼政治委员。军事工程学院创建时,毛泽东主席亲自为学院颁发《训词》,为院刊题写刊名“工学”。1970年学院主体南迁长沙,改名为长沙工学院。1978年,学校在邓小平主席的直接关怀下改建为国防科学技术大学。1999年,江泽民主席签署命令组建新的国防科学技术大学,并于2003年为学校题写“厚德博学、强军兴国”校训。2007年,胡锦涛主席勉励学校为推进科技强军战略、建设创新型国家作出新的更大贡献。2013年11月5日,习近平主席亲临学校视察并发表重要讲话,发出“加快建设具有我军特色的世界一流大学,努力把国防科学技术大学办成高素质新型军事人才培养高地、国防科技自主创新高地”的伟大号召。2017年,学校以原国防科学技术大学、国际关系学院、国防信息学院、西安通信学院、电子工程学院以及理工大学气象海洋学院为基础重建,校本部设在长沙。2017年7月19日,习近平主席为新组建的国防科技大学授军旗、致训词。习近平主席指出:“国防科技大学是高素质新型军事人才培养和国防科技自主创新高地。要紧跟世界军事科技发展潮流,适应打赢信息化局部战争要求,抓好通用专业人才和联合作战保障人才培养,加强核心关键技术攻关,努力建设世界一流高等教育院校”。

国防科技大学计算机科学与技术、软件工程、信息通信与工程、航天宇航科学与技术和管理科学与工程被列为“双一流”建设学科(见“双一流”建设高校及建设学科名单\_中国教育在线)。同时,在2017年的第四轮学科评估中,计算机科学与技术A+、软件工程A+、信息通信与工程、控制科学与工程、光学工程、机械工程、管理科学与工程均为A类学科(见全国第四轮学科评估结果公布)。计算机科学与技术、软件工程、管理科学与工程等学科在2017年公布的学科评估中均为A+学科,且均为国家双一流重点建设学科。

报考国防科技大学无军籍地方硕士和地方博士生的理由:

(1)免报名费、免住宿费(博士为1-2人间,非上下铺、带公用厨房),健身场馆(如健身房、游泳馆、网球场、篮球场、羽毛球场等)应有尽有,且供学生免费使用。

(2)所有地方硕士生和地方博士生就学学年的学费全免。

(3)地方硕士生补助每个月>3500元(含助学金),地方博士生科研补助>4500元/月(含助学金),且发放年限为所有就读年限,万一延期毕业依旧有科研补助。

(4)书籍购置费用全报销,硬件配置等条件优越,出差补助150元/天,出差住宿标准为350-500元。

(5)全国师生比最高的高校,硕士生和博士生主要是完成学术论文,不要求博士参与完成无意义的工程项目开发。

\end{cabstract}
\ckeywords{国防科技大学; “双一流”A类; 985; 211; 哈军工}

\begin{eabstract}
National University of Defense Technology is a comprehensive national key university based in Changsha, %
Hunan Province, China. It is under the dual supervision of the Ministry of National Defense %
and the Ministry of Education, designated for Project 211 and Project 985, %
the two national plans for facilitating the development of Chinese higher education. %

NUDT was originally founded in 1953 as the Military Academy of Engineering in Harbin of Heilongjiang Province. %
In 1970 the Academy of Engineering moved southwards to Changsha and was renamed Changsha Institute of Technology.%
 The Institute changed its name to National University of Defense Technology in 1978.

\end{eabstract}
\ekeywords{NUDT; 985; 211; ME}


\input{data/denotation}

%</thesis>
%    \end{macrocode}
%
%<thesis>%书写正文,可以根据需要增添章节; 正文还包括致谢,参考文献与成果
%\changes{v1.4}{2009/10/31}{将成果移动到参考文献之后}
%
%    \begin{macrocode}
%<*thesis>
\mainmatter
\input{data/chap01}
\input{data/chap02}

\input{data/ack}

%</thesis>
%    \end{macrocode}
%
% 在\LaTeX{}下管理参考文献将极其方便,建议使用Jabref生成条目,
% 用\verb|\cite|(其中\verb|upcite|是上标索引)索引即可。
% \verb|refs.bib|是你的参考文献名。
%    \begin{macrocode}
%<*thesis>
\cleardoublepage
\phantomsection
\addcontentsline{toc}{chapter}{参考文献}
\bibliographystyle{bstutf8}
\bibliography{ref/refs}

\input{data/resume}
%</thesis>
%    \end{macrocode}
%
%<thesis>% 最后,需要的话还要生成附录,全文随之结束。
%    \begin{macrocode}
%<*thesis>
\appendix
\backmatter
\input{data/appendix01}

\end{document}
%</thesis>
%    \end{macrocode}
%
% 当然还有一些收尾工作,校验审阅自不必说。接下来你需要:修改论文中英文日期,
% 生成盲评,生成明(盲)评A3封面。
%
% {\color{blue}Happy \TeX{}ing! 欢迎提各式各样的意见!}
%
% \newpage\relax%
%
% \StopEventually{\PrintChanges}
% \clearpage
%
% \section{实现细节}
% 我们首先介绍文档模板的基本信息以及宏包和配置,
% 然后依照国防科技大学论文模板的书写规范一节一节的介绍实现步骤。
%
% \changes{v1.2}{2009/09/28}{添加了A3封面制作}
%
% \subsection{基本信息}
%    \begin{macrocode}
%<cls>\NeedsTeXFormat{LaTeX2e}[1999/12/01]
%<cls>\ProvidesClass{nudtpaper}
%<cfg>\ProvidesFile{nudtpaper.cfg}
%<cls|cfg>[2017/06/15 v2.7 NUDT paper template]
%    \end{macrocode}
%
% \subsection{宏包配置}
%
%<*cls>
%
%\changes{v0.99}{2009/08/17}{add package options}
% 当前的宏包选项在之前已经介绍了,下面是实现步骤,就是几个\verb|if|。
%\changes{v1.6}{2009/12/01}{添加单独的单双面控制}
%\changes{v2.0}{2010/11/09}{添加盲评控制}
%
%    \begin{macrocode}
\newif\ifismaster\ismastertrue
\DeclareOption{master}{\ismastertrue}
\DeclareOption{doctor}{\ismasterfalse}
\newif\ifisanon\isanonfalse
\DeclareOption{anon}{\isanontrue}
\newif\ifistwoside\istwosidefalse
\DeclareOption{twoside}{\istwosidetrue}
\DeclareOption*{\PackageWarning{nudtpaper}{Unknown Option '\CurrentOption'}}
% handle fonts
\newif\ifisttf\isttffalse
\newif\ifisotf\isotffalse
\newif\ifisfz\isfzfalse
\newif\ifisfandol\isfandolfalse
\DeclareOption{ttf}{\isttftrue}
\DeclareOption{otf}{\isotftrue}
\DeclareOption{fz}{\isfztrue}
\DeclareOption{fandol}{\isfandoltrue}
\ProcessOptions\relax
%    \end{macrocode}
%
% 首先调用在文档类书写中需要的过程控制语句,在计算一些\verb|length|时要用到
%    \begin{macrocode}
\RequirePackage{ifthen,calc}
%    \end{macrocode}
%
% 接着我们导入文本类,该模板基于标准的书籍模板book,其默认格式为单面打印。
% 博士论文如需双面打印,必须指定\verb|twoside|选项。双开的含义是章节总是
% 起在右手边,左手空白页为完全的空白页,不包含页眉页脚。
%
% \changes{v1.6}{2009/12/01}{修改开关选项}
%
%    \begin{macrocode}
\ifistwoside
  \LoadClass[a4paper,12pt,openright,twoside]{book}
\else
  \LoadClass[a4paper,12pt,openany]{book}
\fi
%    \end{macrocode}
%
% 我们直接用\textsf{geometry}宏包进行页面边距的设定,调用titlesec设定标题以及页眉页脚,
% 用\textsf{titletoc}设定目录格式。需要改动的可以参考这三个宏包的说明文档。
%
%    \begin{macrocode}
\RequirePackage[includeheadfoot]{geometry}
\RequirePackage[center,pagestyles]{titlesec}
\RequirePackage{titletoc}
%    \end{macrocode}
%
% 文档中另外重要的两个部分是表格和图片。
% 首先来看图片:\textsf{graphicx}宏包是必不可少的,
% 并排图形。\textsf{subfigure} 已经不再推荐,用新的 \textsf{subfig}。
% 加入 \verb|config| 选项
% 以便兼容 \textsf{subfigure} 的命令。浮动图形和表格标题样式。\textsf{caption2} 已经不
% 推荐使用,采用新的 \textsf{caption}。它会自动被 \textsf{subfig} 装载进来。所以可以在
% 后面使用 \textbf{captionsetup} 命令,宏包\textsf{float}的作用是可以用H命令,
% 将浮动对象强制放在这里(副作用是版面可能不好):
%
%    \begin{macrocode}
\RequirePackage{graphicx}
\RequirePackage[config]{subfig}
\RequirePackage{float}
%    \end{macrocode}
%
% 再来看表格:我们采用\textsf{longtable}来处理长的表格,还需要\textsf{array}包;
% 标准的论文需要表格为三线表,这里引用\textsf{booktabs}宏包来处理,
% 这样,我们就可以简单的使用\verb|\toprule|,\verb|\midrule|,\verb|bottomrulle|
% 这样的命令;
% 为了在表格中支持跨行,需要引入\textsf{multirow}包,\textsf{tabularx}的作用是为了使用
% 固定宽度的表格,\textsf{slashbox}可以让我们在表格中使用反斜线:
%    \begin{macrocode}
\RequirePackage{array}
\RequirePackage{longtable}
\RequirePackage{booktabs}
\RequirePackage{multirow}
\RequirePackage{tabularx}
\RequirePackage{slashbox}
%    \end{macrocode}
% 表格和图片的例子可以搜索C\TeX{}论坛或者看示例文件。
%
% 引入\textsf{paralist}来达到比较好看的列表环境
%    \begin{macrocode}
\RequirePackage[neverdecrease]{paralist}
%    \end{macrocode}
%
% 文档中还需要一定的色彩控制和字体控制
%    \begin{macrocode}
\RequirePackage{xcolor}
%    \end{macrocode}
%
% 为了排出漂亮的数学公式,\textsf{amsmath}包是必不可少的,
% 需要注意的是,新版本的论文模不再使用\textsf{txfonts}宏包,
% 为了支持希腊正体字母,需要调用\verb|upgreek.sty|,使用方法是\verb|\up<greek>|。
% 注意到这个宏包前面加上了\verb|Symbolsmallscale|选项,这是为了调整希腊字母的大小而设定的。
% 如果用户不满意这个宏包的积分号
% 等符号,倾向与使用传统的\LaTeX{}风格的数学符号,那么可以使用
% \textsf{mathptmx}宏包,但要把\verb|upgreek|的选项改为\verb|Symbol|,要不然
% 正体希腊字母要显得比正常字符小一点哦。
% 而大写斜体希腊字母(变量)可以通过\textsf{amsmath}的\verb|\var<Greek>|得到。
% 对于希腊字母的加粗使用\verb|bm|宏包,而一般变量的加粗那就使用\verb|\mathbf|吧!
% \changes{v2.0}{2010/11/09}{去掉fontspec,传递no-math到xeCJK,加入bm宏包}
% \changes{v2.2}{2011/07/16}{去掉txfonts宏包,使用lm字体,添加svgreek.sty}
% \changes{v2.2}{2011/07/16}{修改,仍旧使用upgreek, mathptmx, bm组合}
% \changes{v2.2}{2011/09/25}{修改,使用upgreek, txfonts, bm组合}
% \changes{v2.2}{2012/11/28}{给用户提供额外的选项,还是mtpro比较漂亮}
% \changes{v2.3}{2013/12/27}{调整数学字体,加入mtpro使用说明,并且仿照IEEE模板,修改displaypenalty}
% \changes{v2.5}{2015/05/11}{去掉txfonts,用默认数学字体}
%    \begin{macrocode}
\RequirePackage{amsmath,amssymb}
\RequirePackage[Symbolsmallscale]{upgreek}
%\RequirePackage{amsmath}
%\RequirePackage[amsbb,eufrak,compatiblegreek,subscriptcorrection,nofontinfo]{mtpro2}
\interdisplaylinepenalty=2500
\RequirePackage{bm}
\RequirePackage[T1]{fontenc}
\RequirePackage[amsmath,thmmarks,hyperref]{ntheorem}
%    \end{macrocode}
% 需要注意的是,如果用户有\verb|mtpro2|包,还是强烈建议使用这个的,因为数学公式
% 在这个包下显得特别的美观。虽然下载和安装不属于这篇使用说明的范畴,但是,上面的注释部分
% 可以给大家如何使用的一个简单的例子。当你安装好\verb|mtpro2|之后,主要取消注释,并且将
% 上面的三个包注释掉即可。
%
% 本文档类直接采用\XeTeX{}引擎,方便了字体配置以及编译,
% 这里需要调用\textsf{XeCJK}宏包,no--math的作用是不改变先前数学宏包设定的数学字体。
% 同时采用\textsf{indentfirst}宏包管理文字的缩进:
% \changes{v1.8}{2010/10/15}{修改了默认的xeCJK的选项,为了兼容旧的xeCJK版本,normalindentfirst选项暂不使用,而是在后面添加indentfirst包}
% \changes{v2.0}{2010/11/10}{传递no-math给xeCJK里面的fontspec宏包}
% \changes{v2.2}{2011/07/03}{移除CJKtextspace, CJKmathspace, CJKnumber选项}
% \changes{v2.4}{2015/02/09}{移除CJKnumber选项,添加新的CJKnumb宏包。旧版本TexLive无需更改。}
% \changes{v2.6}{2017/05/15}{移除CJKnumb宏包,统一使用zhnumber宏包。推荐使用2016以后的TexLive版本。}
%
%    \begin{macrocode}
\RequirePackage[CJKchecksingle,no-math]{xeCJK}
\RequirePackage{zhnumber}
\RequirePackage{indentfirst}
%    \end{macrocode}
%
% 另外一个关键部分是文献索引,包括书签以及参考文献的索引,记得\textsf{hyperref}配合
% \XeTeX{}使用时暂不能开启Unicode选项,新的发行版已经移除\textsf{hypernat}包。
% 另外还要注意,你最终的打印版肯定不希望有花花绿绿的链接对吧?
% 那就把下面那行\verb|hyperref|注释掉就行了或者把选项改为\verb|\colorlinks=false|即可。
% \changes{v2.1}{2010/12/29}{移除hypernat包}
% \changes{v2.2}{2011/07/17}{移除hyperref的CJKbookmarks旋向}
% \changes{v2.3}{2013/12/27}{加入色彩版hyperref}
% \changes{v2.5}{2015/05/11}{提示用户在最终打印版时去掉链接颜色}
%    \begin{macrocode}
\RequirePackage[numbers,sort&compress,square]{natbib}
\RequirePackage[colorlinks=true,linkcolor=blue,citecolor=red,pdfborder=0 1 1]{hyperref}
\RequirePackage{datetime}
%\RequirePackage[pdfborder=0 0 1]{hyperref}
%    \end{macrocode}
%</cls>
%
%\subsection{基础配置}
% 本章主要介绍模板中用到的基本的元素和定义,现在包括两部分: 字体,字号和字体命令
%
%\subsubsection{字体定义}
% 我们首先来处理\TeX{}中最令人棘手的字体问题,
% 在使用\textsf{XeCJK}包之后,配置和选择很容易,
% 预先设定好一些字体命令是为了后面方便的更改文本字体的需要。
% 首先我们开启\TeX{}连字符:
%    \begin{macrocode}
%<*cls>
\defaultfontfeatures{Mapping=tex-text}
%</cls>
%    \end{macrocode}
%
% 之后用\textsc{XeCJK}包提供的命令设定字体,用户可以选择使用TTF还是OTF字体,
% Adobe的OpenType字体在排版上更具备优势,文档显示锐利,推荐使用。
% 另外在这一个新版本中,我们推荐用户也可以使用方正的字体,只要使用\verb|FZ|选项即可。
% 中注释掉相关的字体就可以。方正字体的有点是标点符号的位置无需修正,且字体之间配合很好。
% \verb|setcharclass|的作用是纠正xunicode、xeCJK的一些设定:
%
% \changes{v0.99}{2009/08/17}{add options TTF and OTF}
% \changes{v1.9}{2010/10/28}{定义一个cusong字体,使用的是中宋}
% \changes{v2.3}{2013/12/27}{用户可以考虑使用方正字体,加入FZ选项}
% \changes{v2.5}{2015/05/11}{重新更改字体选项的调用方式,现在变成三个ttf、otf、fz}
% \changes{v2.5}{2015/05/11}{这种方式用户可以特别方便的添加自己的字体集}
%
%    \begin{macrocode}
%<*cls>
\xeCJKsetcharclass{"0}{"2E7F}{0}
\xeCJKsetcharclass{"2E80}{"FFFF}{1}

% ZhongYi 中易字体
\newcommand{\installttf}{
    %%%% Windows Thesis Fonts
    \setmainfont{Times New Roman PS Std}
    \setsansfont{Arial}
    \setmonofont{Courier New}
    %%%% Using Office Family Fonts
    \setCJKmainfont[BoldFont={STZhongsong}]{SimSun}
    \setCJKsansfont{SimHei} % Hei
    \setCJKmonofont{FangSong} % Fangsong
    %%%% alias
    \setCJKfamilyfont{song}{SimSun}
    \setCJKfamilyfont{hei}{SimHei}
    \setCJKfamilyfont{fs}{FangSong} % fang-song
    \setCJKfamilyfont{kai}{KaiTi} % Kai
}

% Adobe 字体
\newcommand{\installotf}{
    %%%% Windows Thesis Fonts
    \setmainfont{Times New Roman PS Std}
    \setsansfont{Arial}
    \setmonofont{Courier New}
    %%%% Using Adobe Family Fonts
    \setCJKmainfont[BoldFont={STZhongsong}]{Adobe Song Std}
    \setCJKsansfont{Adobe Heiti Std} % Hei
    \setCJKmonofont{Adobe Fangsong Std} % Fangsong
    %%%% alias
    \setCJKfamilyfont{song}{Adobe Song Std}
    \setCJKfamilyfont{hei}{Adobe Heiti Std}
    \setCJKfamilyfont{fs}{Adobe Fangsong Std} % fang-song
    \setCJKfamilyfont{kai}{Adobe Kaiti Std} % Kai
}

% fz 方正字体 [recommended]
\newcommand{\installfz}{
    %%%% Windows Thesis Fonts
    \setmainfont{Times New Roman PS Std}
    \setsansfont{Arial}
    \setmonofont{Courier New}
    %%%% Using Founder Family Fonts
    \setCJKmainfont[BoldFont={FZYaSong-DB-GBK}]{FZShuSong_GB18030-Z01}
    \setCJKsansfont{FZHei-B01} % Hei
    \setCJKmonofont{FZFangSong-Z02} % fs
    %%%% alias
    \setCJKfamilyfont{song}{FZShuSong_GB18030-Z01}
    \setCJKfamilyfont{hei}{FZHei-B01}
    \setCJKfamilyfont{fs}{FZFangSong-Z02} % fang-song
    \setCJKfamilyfont{kai}{FZKai-Z03} % Kai
}

% fandol [incomplete in 2015]
\newcommand{\installfandol}{
    %%%% Windows Thesis Fonts
    \setmainfont{Times New Roman PS Std}
    \setsansfont{Arial}
    \setmonofont{Courier New}
    %%%% Using Fandol Family Fonts
    \setCJKmainfont{FandolSong}
    \setCJKsansfont{FandolHei} % Hei
    \setCJKmonofont{FandolFang} % fs
    %%%% alias
    \setCJKfamilyfont{song}{FandolSong}
    \setCJKfamilyfont{hei}{FandolHei}
    \setCJKfamilyfont{fs}{FandolFang} % fang-song
    \setCJKfamilyfont{kai}{FandolKai} % Kai
}

\ifisttf
\installttf
\fi

\ifisotf
\installotf
\fi

\ifisfz
\installfz
\fi

\ifisfandol
\installfandol
\fi

%</cls>
%    \end{macrocode}
%
% \changes{v1.6}{2009/12/01}{替换OTF英文字体为标准Windows自带字体}
% \changes{v2.3}{2013/12/27}{添加FZ字体选项}
%
% 选定好字体之后,就是设定字体别名,这样我们就可以在文档的其他部分直接使用较短的命令来
% 指定特定的字体了:
%
%    \begin{macrocode}
%<*cls>
% command alias
\newcommand{\cusong}{\bfseries}
\newcommand{\song}{\CJKfamily{song}}     % 宋体
\newcommand{\fs}{\CJKfamily{fs}}         % 仿宋体
\newcommand{\kai}{\CJKfamily{kai}}       % 楷体
\newcommand{\hei}{\CJKfamily{hei}}       % 黑体
\def\songti{\song}
\def\fangsong{\fs}
\def\kaishu{\kai}
\def\heiti{\hei}
%</cls>
%    \end{macrocode}
%
% \subsubsection{字号定义}
%下面就是定义字号大小,这一部分我们有两个参考,其一是:
%
% \begin{verbatim}
% 参考科学出版社编写的《著译编辑手册》(1994年)
% 七号      5.25pt       1.845mm
% 六号      7.875pt      2.768mm
% 小五      9pt          3.163mm
% 五号      10.5pt       3.69mm
% 小四      12pt         4.2175mm
% 四号      13.75pt      4.83mm
% 三号      15.75pt      5.53mm
% 二号      21pt         7.38mm
% 一号      27.5pt       9.48mm
% 小初      36pt         12.65mm
% 初号      42pt         14.76mm
%
% 这里的 pt 对应的是 1/72.27 inch,也就是 TeX 中的标准 pt
% \end{verbatim}
%
% 另外一个来自WORD中的设定:
% \begin{verbatim}
% 初号 = 42bp = 14.82mm = 42.1575pt
% 小初 = 36bp = 12.70mm = 36.135 pt
% 一号 = 26bp = 9.17mm = 26.0975pt
% 小一 = 24bp = 8.47mm = 24.09pt
% 二号 = 22bp = 7.76mm = 22.0825pt
% 小二 = 18bp = 6.35mm = 18.0675pt
% 三号 = 16bp = 5.64mm = 16.06pt
% 小三 = 15bp = 5.29mm = 15.05625pt
% 四号 = 14bp = 4.94mm = 14.0525pt
% 小四 = 12bp = 4.23mm = 12.045pt
% 五号 = 10.5bp = 3.70mm = 10.59375pt
% 小五 = 9bp = 3.18mm = 9.03375pt
% 六号 = 7.5bp = 2.56mm
% 小六 = 6.5bp = 2.29mm
% 七号 = 5.5bp = 1.94mm
% 八号 = 5bp = 1.76mm
%
% 1bp = 72.27/72 pt
% \end{verbatim}
%
% 我们采用习惯的字号设定方法(也就是WORD中的设定),首先编写字体设置命令:
%
%\begin{macro}{\choosefont}
% 我们可以使用 |\choosefont| 来选择字体, 字体设定这些大多是从清华的模板拷过来的。
%
%    \begin{macrocode}
%<*cls>
\newlength\thu@linespace
\newcommand{\thu@choosefont}[2]{%
    \setlength{\thu@linespace}{#2*\real{#1}}%
    \fontsize{#2}{\thu@linespace}\selectfont}
\def\thu@define@fontsize#1#2{%
    \expandafter\newcommand\csname #1\endcsname[1][\baselinestretch]{%
    \thu@choosefont{##1}{#2}}}
%</cls>
%    \end{macrocode}
%\end{macro}
%
%设定具体的字体大小:
%
%    \begin{macrocode}
%<*cls>
\thu@define@fontsize{chuhao}{42bp}
\thu@define@fontsize{xiaochu}{36bp}
\thu@define@fontsize{yihao}{26bp}
\thu@define@fontsize{xiaoyi}{24bp}
\thu@define@fontsize{erhao}{22bp}
\thu@define@fontsize{xiaoer}{18bp}
\thu@define@fontsize{sanhao}{16bp}
\thu@define@fontsize{xiaosan}{15bp}
\thu@define@fontsize{sihao}{14bp}
\thu@define@fontsize{banxiaosi}{13bp}
\thu@define@fontsize{xiaosi}{12bp}
\thu@define@fontsize{dawu}{11bp}
\thu@define@fontsize{wuhao}{10.5bp}
\thu@define@fontsize{xiaowu}{9bp}
\thu@define@fontsize{liuhao}{7.5bp}
\thu@define@fontsize{xiaoliu}{6.5bp}
\thu@define@fontsize{qihao}{5.5bp}
\thu@define@fontsize{bahao}{5bp}
%</cls>
%    \end{macrocode}
%
%\subsubsection{自定命令}
% 有一些常量,测试,自定义的命令等都放在这里,待到论文逐渐完善之后再做定夺,
% 当然用户自己的命令也可以在此添加,事实上如果natbib传递的是superscript,
% \verb|cite|命令默认就成了上标了。这里不加入这个选项,而是单独编写一个命令:
%
%    \begin{macrocode}
%<*cls>
\newcommand{\upcite}[1]{\textsuperscript{\cite{#1}}} % 上标形式引用
\newcommand{\china}{中华人民共和国}
\def\nudtpaper{\textsc{Nudt}\textsc{Paper}}
\newcommand{\pozhehao}{\kern0.2em\rule[0.8ex]{1.6em}{0.1ex}\kern0.2em}
\newcommand{\xiaopozhe}{\kern0.2em\rule[0.8ex]{0.6em}{0.1ex}\kern0.2em}
%</cls>
%    \end{macrocode}
% \changes{v2.5}{2015/05/11}{添加了一个xiaopozhe命令,用作中文连词符}
%
%\subsubsection{中文元素}
%
% 默认的页面元素的英文名,诸如Contents为目录,Abstract为摘要等,
% 我们首先将他们一一中文化:
% \changes{v0.992}{2009/08/19}{修改图表编号格式}
% \changes{v1.3}{2009/10/14}{修改图目录和表目录}
% \changes{v2.6}{2017/05/15}{更新zhnumber宏包兼容命令}
% \changes{v2.6}{2017/05/15}{重定义zhnumber宏包中的zhtoday命令}
%
%    \begin{macrocode}
%<*cls>
\renewcommand\contentsname{目\hspace{1em}录}
\renewcommand\listfigurename{图\hspace{1em}目\hspace{1em}录}
\renewcommand\listtablename{表\hspace{1em}目\hspace{1em}录}
\newcommand\listequationname{公式索引}
\newcommand\equationname{公式}
\renewcommand\bibname{参考文献}
\renewcommand\indexname{索引}
\renewcommand\figurename{图}
\renewcommand\tablename{表}
\renewcommand\appendixname{附录}
\def\CJK@today{\zhdigits{\the\year}年\zhnumber{\the\month}月}
\renewcommand\zhtoday{\CJK@today}
\newcommand\entoday{\monthname{}, \the\year}
%</cls>
%    \end{macrocode}
%
% 好,下面就开始按照论文模板要求进行排版!
%
%\subsection{编写要求}
% 学校规定,论文需采用白色纸双面打印。
% 学位论文用A4($210mm\times{}297mm$)标准大小的白纸,
% 在打字或印刷时,要求纸的四周留足空白边缘,以便装订、复制和读者批注。
% 每一面的上方(天头)和下方(地角)分别留边25mm,左侧(订口)
% 和右侧(切口)分别留边30mm,页眉与页脚分别为23mm。
%
% 实现起来很简单,只要调用\textsf{geometry}的版面控制命令即可,
% 方法为先把word模板转化为PDF,
% 用Adobe的裁剪功能查看页边距,进行微调,直到比对正确为止,设定如下:
%
% \changes{v0.991}{2009/08/18}{modify bottom skip}
% \changes{v1.1}{2009/09/26}{修改footskip容限以及bottom的值,为了容下longtab的''下一页''}
% \changes{v1.4}{2009/10/28}{减小页眉skip 1mm,用word叠印}
% \changes{v1.4}{2009/10/30}{增大页眉sep .5mm,用word叠印}
%
%    \begin{macrocode}
%<*cls>
\geometry{top=21mm,bottom=25.5mm,left=30mm,right=30mm}
\geometry{headheight=9mm,headsep=1mm,footskip=9mm}
%</cls>
%    \end{macrocode}
%
%\subsection{页眉页脚}
%
% 我们采用titlesec进行页面配置。
% 页面中的主要元素有Chapter,Section,Subsection等元素的外观,
% 位置,颜色字体等,页面元素还包括页眉页脚。这种方法配置简便,易管理。
% 国防科大的论文需要在页眉处画两根横线,我们通过下面的命令实现:
%
%\begin{macro}{\setheadrule}
% 这个命令属于更改\textsf{titlesec}中的一个画页眉的命令,稍加调整:
% \changes{v0.991}{2009/08/18}{modify headrull, s.t. all geometry match}
% \changes{v1.9}{2010/10/28}{去掉headsep,修改headrule,在sethead后添加raisebox}
%
%    \begin{macrocode}
%<*cls>
\renewcommand\setheadrule[1]{%
  \ifdim#1=\z@
    \let\makeheadrule\@empty
  \else
    \def\makeheadrule{%
    \makebox[0pt][l]{\rule[.2\baselineskip]{\linewidth}{1.5pt}}%
    \rule{\linewidth}{1.5pt}}%
  \fi}
%</cls>
%    \end{macrocode}
%\end{macro}
%
% 由于Chapter第一页默认是\verb|plain|页面格式,
% 章节的其余部分是在Matter中设定的页面格式,为了简单起见,
% 我们就直接更改\verb|plain|页面设置,
% 要求为5号宋体居中放置,画页眉页脚,页脚为1磅黑线
%
% \changes{v0.992}{2009/08/20}{renewpagestyle里面前导的空格可能导致clearpage生成新的一页,将空格去掉}
% \changes{v0.993}{2009/08/26}{修改标题,博士硕士对应不同的页眉}
%
%    \begin{macrocode}
%<*cls>
\renewpagestyle{plain}{
\sethead{}{\raisebox{.65\baselineskip}{\songti \wuhao \ifisanon{~}\else{国防科技大学研究生院\@optionpaperclass{}学位论文}\fi}}{}%
\setfoot{}{{\songti \wuhao 第~\thepage~页}}{}%
\headrule%
\footrule%
}
\setfootrule{1bp}
%</cls>
%    \end{macrocode}
%
%\subsection{编写格式}
%
% 当页面设置好之后,就是在论文的不同部分分别调用,一般来说论文类的书籍
% 分为三个matter,为前言区(前置部分),正文区(主体),后文区(附录),
% 在国防科大论文书写要求中,
% 需要将摘要单独进行页码编号,其编号为小写罗马字母,为此,
% 可以将摘要单独设定为一个matter,
% 名叫就叫做MidMatter,称作摘要区。每个Matter我们都一一介绍。
%
% 首先看前置部分,主要包括封面,目录,摘要等,实现为:
%
%    \begin{macrocode}
%<*cls>
\renewcommand\frontmatter{%
    \if@openright\cleardoublepage\else\clearpage\fi
    \@mainmatterfalse
    \pagenumbering{Roman}
    \pagestyle{plain}}
\newcommand\midmatter{%
    \if@openright\cleardoublepage\else\clearpage\fi
    \@mainmatterfalse
    \pagenumbering{roman}
    \pagestyle{plain}}
%</cls>
%    \end{macrocode}
%
% 之后为文章的正文区,采用阿拉伯数字编页码:
%
%    \begin{macrocode}
%<*cls>
\renewcommand\mainmatter{%
    \if@openright\cleardoublepage\else\clearpage\fi
    \@mainmattertrue
    \pagenumbering{arabic}
    \pagestyle{plain}}
%</cls>
%    \end{macrocode}
%
% 最后是附录部分,由于他的章节标题与正文中不一样(不是第几章,而是附录几),
% 我们需要单独设定:
%
%    \begin{macrocode}
%<*cls>
\renewcommand\backmatter{%
    \if@openright\cleardoublepage\else\clearpage\fi
    \titleformat{\chapter}{\filcenter \heiti \sanhao}{附录\,\thechapter\,}{1em}{}
    \titlecontents{chapter}[0pt]{\vspace{0.25\baselineskip} \heiti \xiaosi[1.25]}
      {附录\,\thecontentslabel\quad}{}
      {\hspace{.5em}\titlerule*{.}\contentspage}
    \@mainmattertrue
    \pagestyle{plain}}
%</cls>
%    \end{macrocode}
%
% 我们重新定义\verb|cleardoublepage|,使得生成完全的空白页,页面模式为\verb|empty|
%    \begin{macrocode}
%<*cls>
\renewcommand\cleardoublepage{\clearpage\if@openright \ifodd\c@page\else
  \newpage{}
  \thispagestyle{empty}
  \vspace*{\fill}
  \begin{center}
  \end{center}
  \vspace*{\fill}
  \clearpage\fi\fi%
}
%</cls>
%    \end{macrocode}
%
%\subsubsection{前置目录}
% 前置部分的封面在后面详细介绍。首先看目录,要求为:
% 目次页由论文的章、节、条、项、附录等的序号、名称和页码组成,
% 另页排在序之后。目次页标注学位论文的前三级目录。
% 标题统一用“目录”,黑体3字号字居中,段前、段后间距为1行;
% 各章(一级目录)名称用黑体小4号字,段前间距为0.5行,
% 段后间距为0行; 其它(二、三级目录)用宋体小4号字,
% 段前、段后间距为0行。:
%
% 在\LaTeX{}中,Chapter在目录中默认是没有点的,我们加上,另外我们一并将
% 目录中的section和subsection设定好,
% \changes{v0.991}{2009/08/18}{modify TOC baselineskip and font lineskip to 1.25}
% \changes{v2.6}{2017/05/15}{更新使用zhnumber生成中文章节编号}
%
%    \begin{macrocode}
%<*cls>
\titlecontents{chapter}[0pt]{\vspace{0.25\baselineskip} \heiti \xiaosi[1.25]}
    {第\zhnumber{\thecontentslabel}章\quad}{}
    {\hspace{.5em}\titlerule*{.}\contentspage}
\titlecontents{section}[2em]{\songti \xiaosi[1.25]}
    {\thecontentslabel\quad}{}
    {\hspace{.5em}\titlerule*{.}\contentspage}
\titlecontents{subsection}[4em]{\songti \xiaosi[1.25]}
    {\thecontentslabel\quad}{}
    {\hspace{.5em}\titlerule*{.}\contentspage}
%</cls>
%    \end{macrocode}
%
% 然后是表目录和图目录,内容用宋体小4号字,在同学使用模板时,需要标题对齐,
% 我们一并在这里实现:
% \changes{v0.993}{2009/08/25}{添加makebox使得图表标题对齐}
%
%    \begin{macrocode}
%<*cls>
\titlecontents{figure}[0pt]{\songti \xiaosi[1.25]}
    {\makebox[3.5em][l]{图~\thecontentslabel\quad}}{}
    {\hspace{.5em}\titlerule*{.}\contentspage}
\titlecontents{table}[0pt]{\songti \xiaosi[1.25]}
    {\makebox[3.5em][l]{表~\thecontentslabel\quad}}{}
    {\hspace{.5em}\titlerule*{.}\contentspage}
%</cls>
%    \end{macrocode}
%
% 书籍模板中,在LOF或者LOT章节之间会默认插入额外的距离,我们通过修改下面这个命令移除,
% 这个方法不是一个完美的办法,\textbf{注意}:下面的代码不要去深究或者理解,
% 这只是把book.cls中的内容复制过来,然后去掉包含addvspace命令的两行。
% 我实在找不出更加好的办法,如果你有,可以联系我。
%
% \changes{v0.993}{2009/08/25}{移除LOF及LOT中章节之间额外的距离}
%
%    \begin{macrocode}
%<*cls>
\renewcommand\chapter{\if@openright\cleardoublepage\else\clearpage\fi
                    \thispagestyle{plain}%
                    \global\@topnum\z@
                    \@afterindentfalse
                    \secdef\nudt@chapter\@schapter}
\def\nudt@chapter[#1]#2{
  \ifnum \c@secnumdepth >\m@ne
    \if@openright\cleardoublepage\else\clearpage\fi
    \phantomsection
    \if@mainmatter
      \refstepcounter{chapter}%
      \addcontentsline{toc}{chapter}%
        {\protect\numberline{\thechapter}#1}%
    \else
      \addcontentsline{toc}{chapter}{#1}%
    \fi
  \else
    \addcontentsline{toc}{chapter}{#1}%
  \fi
  \chaptermark{#1}%
  \if@twocolumn
    \@topnewpage[\@makechapterhead{#2}]%
  \else
    \@makechapterhead{#2}%
    \@afterheading
  \fi
}
%</cls>
%    \end{macrocode}
%
%\subsubsection{前置摘要}
%
% 摘要的要求为题目黑体3字号字居中,段前、段后间距为1行,内容用宋体小4号字,
% 英文摘要内容用Time New Roman小4号字。
% 中文关键字以黑体小4号字另起一行,排在摘要的下方,英文关键字用Arial小4号字。
%
% \changes{v1.8}{2010/10/15}{ABSTRACT和英文关键字需要用Arial字体}
% \changes{v2.5}{2015/05/11}{英文关键词也要加粗}
%    \begin{macrocode}
%<*cls>
\newcommand\cabstractname{摘\hspace{1em}要}
\newcommand\eabstractname{ABSTRACT}
\newcommand\ckeywordsname{关键词}
\newcommand\ckeywords[1]{{\hei\xiaosi \ckeywordsname: #1}}
\newcommand\ekeywordsname{Key Words}
\newcommand\ekeywords[1]{\textbf{\textsf{\xiaosi \ekeywordsname: #1}}}
\newenvironment{cabstract}{%
    \chapter{\cabstractname}
    \xiaosi
    \@afterheading}
    {\par\vspace{2em}\par}
\newenvironment{eabstract}{%
    \chapter{\textsf{\eabstractname}}
    \xiaosi
    \@afterheading}
    {\par\vspace{2em}\par}
%</cls>
%    \end{macrocode}
%
%\subsection{主体部分}
%
% \subsubsection{标题格式}
% 要求为:
% \begin{compactenum}
% \item	一级标题(章)用黑体3号字居中,1.25倍行距,段前、段后间距为1行,每一章从新的一页开始;
% \item	二级标题(节)用宋体4号粗体字居中,1.25倍行距,段前、段后间距为1行;
% \item	三级标题用黑体小4号字两端对齐,1.25倍行距,段前、段后间距为1行;
% \item	四级标题用宋体小4号粗体字两端对齐,1.25倍行距,段前间距为0.5行,段后间距为0行;
% \end{compactenum}
%
% \changes{v0.991}{2009/08/18}{按照要求设定标题}
% \changes{v0.992}{2009/08/19}{修改secnumdepth使得subsubsection可用}
% \changes{v1.1}{2009/09/26}{修改Title的spacing为弹性值}
% \changes{v1.2}{2009/10/06}{去掉弹性值,不去生成大量的空白}
% \changes{v1.4}{2009/10/28}{修改chapter段后行距为2ex,段前-1ex,保证上下对称}
% \changes{v1.4}{2009/10/29}{修改chapter段后行距为2.4ex,段前-1.2ex,保证上下对称}
%
% 当章节标题出现的新的一页时,会出现段前距过小的情况,按照milksea的说法是:
% 一般而言,当一个内容在一页开头时,前面的\verb|\vskip|不起作用;
% 类似地,一行开头\verb|\hskip|不起作用。这不是 BUG,如果需要总起效果的间距,
% 用\verb|\vspace*|,文档里面有这样的例子。参照titlesec的文档,需加上:
% \changes{v1.9}{2010/10/28}{增加sectionbreak,设定topskip为0pt}
%
%    \begin{macrocode}
%<*cls>
\newcommand{\sectionbreak}{%
\addpenalty{-300}%
\vspace*{0pt}%
}
\setlength{\topskip}{0pt}
%</cls>
%    \end{macrocode}
% \changes{v1.9}{2010/10/28}{在定义了粗宋字体之后,按照学位论文要求设定标题字体}
% \changes{v1.9}{2010/10/28}{使用了ttltips.pdf的设置chapter距顶端距离的办法}
% \changes{v2.3}{2013/12/27}{修改了footnotesize,当一页中有两个footnote时,间距调整正常}
% \changes{v2.3}{2013/12/27}{修改了titleformat和titlespacing的定制,更改英文字体,修正间距}
% \changes{v2.6}{2017/05/15}{更新使用zhnumber}
%
%    \begin{macrocode}
%<*cls>
\setcounter{secnumdepth}{3}
\setlength{\footnotesep}{2.2ex \@minus 2bp}
\titleformat{\chapter}{\filcenter\sf \heiti\sanhao[1.25]}{第\zhnumber{\thechapter}章\,}{1em}{}
\titleformat{\section}{\filcenter\bfseries \cusong\sihao[1.25]}{\thesection}{1em}{}
\titleformat{\subsection}{\sf \heiti\xiaosi[1.25]}{\thesubsection}{1em}{}
\titleformat{\subsubsection}{\bfseries \cusong\xiaosi[1.25]}{\thesubsubsection}{1em}{}
\titlespacing{\chapter}{0pt}{2.4ex-\topskip-\heightof{A}}{2.4ex \@minus 2bp}
\titlespacing{\section}{0pt}{2ex-\heightof{a}}{2ex \@minus 2bp}
\titlespacing{\subsection}{2em}{2ex \@minus 2bp}{2ex \@minus 2bp}
\titlespacing{\subsubsection}{2em}{1ex \@minus 2bp}{1ex \@minus 2bp}
%</cls>
%    \end{macrocode}
%
% \changes{v2.5}{2015/05/11}{添加了subsubsection下1ex的距离}
%
%\subsubsection{正文字体}
% 首先确定正文中使用的字体,文档要求正文字体为小四,行距为固定值1.25倍,
% 中文字体为宋体,英文为{Times New Roman}
%
%\begin{macro}{\normalsize}
% 我们重新定义 |\normalsize| 来确定文档的正文字体,
% 同时修改正文中公式与文字间的距离:
% \changes{v1.9}{2010/10/28}{在normalsize后面每一行加上\%号来吃掉多余的空格}
% \changes{v2.2}{2011/07/16}{减小公式之间距离rubber space的上界}
% \changes{v2.2}{2011/09/25}{减小公式之间距离rubber space的下界}
% \changes{v2.3}{2013/12/27}{移除了abovedisplay的正向rubberspace}
%
%    \begin{macrocode}
%<*cls>
\renewcommand\normalsize{%
\@setfontsize\normalsize{12bp}{12.87bp}%
\renewcommand{\baselinestretch}{1.3}%
\setlength\abovedisplayskip{10bp \@minus 1bp}%
\setlength\abovedisplayshortskip{10bp \@minus 1bp}%
\setlength\belowdisplayskip{\abovedisplayskip}%
\setlength\belowdisplayshortskip{\abovedisplayshortskip}%
}
%</cls>
%    \end{macrocode}
%\end{macro}
%
% \changes{v0.991}{2009/08/18}{modify normalsize, which will cause headrule shift}
% \changes{v0.991}{2009/08/18}{add comment on displayskip}
% \changes{v1.0}{2009/09/22}{modify display skip}
%
%\subsubsection{正文段落}
% 接下来还有一个细节就是处理段落缩进,文档设定为首行缩进2个字符,
% 这一个命令需要在文档开始时自动执行:
%
% \changes{v1.3}{2009/10/03}{添加checkparameter这一选项,避免由于更新模板导致未定义的情况出现}
% \changes{v1.7}{2010/04/30}{应当在封面制作完后替换tabular}
% \changes{v2.5}{2015/05/12}{删除twochars囧}
%
%    \begin{macrocode}
%<*cls>
\newlength\CJK@twochars
\def\CJKindent{%
  \settowidth\CJK@twochars{中国}%
  \parindent\CJK@twochars}
\AtBeginDocument{%
  \CJKindent\relax
  \checkparameter\relax
}
%</cls>
%    \end{macrocode}
%
% 之后定义段落间距,段前间距以及段后间距都为0
% \changes{v0.993}{2009/08/27}{修改parskip}
% \changes{v2.2}{2011/09/25}{修改parskip,允许少量的调整,1bp}
% \changes{v2.2}{2011/10/14}{修改parskip,仅允许负的少量调整,2bp}
% \changes{v2.3}{2013/12/27}{移除parskip的正向rubberspace}
% \changes{v2.5}{2015/05/11}{减小parskip的负向距离至1bp}
%
%    \begin{macrocode}
%<*cls>
\setlength{\parskip}{0bp \@minus 1bp}
%</cls>
%    \end{macrocode}
%
% 有时候我们需要手动设定字体间距,该命令在声明页使用过:
%\begin{macro}{\ziju}
%    \begin{macrocode}
%<*cls>
\newcommand*{\ziju}[1]{\renewcommand{\CJKglue}{\hskip #1}}
%</cls>
%    \end{macrocode}
%\end{macro}
%
% \changes{v1.4}{2009/10/26}{推荐用户使用紧凑的列表环境}
%
% 这一部分来自Thuthesis的代码,其出发点是不满意\LaTeX{}默认列表环境间距过大,用
% paralist包中的相关环境进行替代。请参考paralist宏包。
%
% \changes{v1.4}{2009/10/26}{修改参考文献的行距设定}
%
% 而同样有间距问题的是参考文献,两个条目之间过大的距离不是很美观,
% 最简单的办法是修改bibsep变量,如果还是不行,我们直接从thuthesis中拿来代码:
%
% \changes{v1.4}{2009/10/26}{修改参考文献的行距}
% \changes{v1.4}{2009/10/29}{修改参考文献左对齐}
% \changes{v2.2}{2011/10/14}{减小文献列表间距,将penalty改为4000}
%
%    \begin{macrocode}
%<*cls>
\renewenvironment{thebibliography}[1]{%
   \chapter*{\bibname}%
   \list{\@biblabel{\@arabic\c@enumiv}}%
        {\renewcommand{\makelabel}[1]{##1\hfill}
         \settowidth\labelwidth{1.1cm}
         \setlength{\labelsep}{0.4em}
         \setlength{\itemindent}{0pt}
         \setlength{\leftmargin}{\labelwidth+\labelsep}
         \addtolength{\itemsep}{-0.7em}
         \usecounter{enumiv}%
         \let\p@enumiv\@empty
         \renewcommand\theenumiv{\@arabic\c@enumiv}}%
    \sloppy\frenchspacing
    \clubpenalty4000%
    \widowpenalty4000%
    \interlinepenalty4000%
    \sfcode`\.\@m}
   {\def\@noitemerr
     {\@latex@warning{Empty `thebibliography' environment}}%
    \endlist\frenchspacing}
%</cls>
%    \end{macrocode}
%
%\subsection{浮动对象}
%
% 浮动对象针对的目标是图片表格,标题为五号字体,
% 图片标题在下,表格标题在上,具体实现为:
% \changes{v1.1}{2009/09/26}{修改float浮动弹性}
% \changes{v2.2}{2011/09/25}{去掉1.0fil改为4bp, 这样不至于生成过大的空白.}
% \changes{v2.3}{2013/12/27}{移除了floatsep所有rubberspace. 原先为正2负2.}
%
%    \begin{macrocode}
%<*cls>
\setlength{\floatsep}{12bp}
\setlength{\intextsep}{12bp}
\setlength{\textfloatsep}{12bp}
\setlength{\@fptop}{0bp}
\setlength{\@fpsep}{12bp}
\setlength{\@fpbot}{0bp}
%</cls>
%    \end{macrocode}
%
% 接下来设置每一页图形占据的比例,这个直接从\thuthesis{}中拿出,
% 具体含义可以参考下面这个网页:
% \url{http://www.ctex.org/documents/latex/graphics/node69.html},
% 里面解释的很清楚,这个布置方法也是网站的推荐:
% \changes{v1.3}{2009/09/29}{调整floatpagefraction的大小}
% \changes{v2.2}{2011/09/10}{重新调整floatpagefraction,使得更为宽松}
% \changes{v2.2}{2011/09/25}{更为宽松的布置条件}
% \changes{v2.2}{2011/09/25}{更为宽松的布置条件,仿照AMSMATH}
% \changes{v2.5}{2015/05/11}{更为宽松的限制,penalty改为5000}
% \changes{v2.5}{2015/05/11}{更为严格的限制,fraction改为0.98}
%
%    \begin{macrocode}
%<*cls>
\renewcommand{\textfraction}{0.01}
\renewcommand{\topfraction}{0.98}
\renewcommand{\bottomfraction}{0.98}
\renewcommand{\floatpagefraction}{0.90}
\clubpenalty=5000
\widowpenalty=5000
\displaywidowpenalty=5000
%</cls>
%    \end{macrocode}
%
% 在修改图片标题距离时,要注意,aboveskip为内距离,也就是标题与浮动体之间的距离,
% belowskip为外距离,也就是标题与正文之间的距离。
% \changes{v1.3}{2009/09/29}{缩小图片标题与下文的距离}
% \changes{v1.7}{2010/04/30}{增添LT array命令,可修改Longtable字体大小}
% \changes{v2.7}{2017/06/15}{{删除图标标题前的多余空格}}
%
%    \begin{macrocode}
%<*cls>
\let\old@tabular\@tabular
\def\thu@tabular{\wuhao[1.25]\old@tabular}
\DeclareCaptionLabelFormat{thu}{{\wuhao[1.25]\song#1~\rmfamily #2}}
\DeclareCaptionLabelSeparator{thu}{\hspace{1em}}
\DeclareCaptionFont{thu}{\wuhao[1.25]}
\captionsetup{labelformat=thu,labelsep=thu,font=thu}
\captionsetup[table]{position=top,belowskip=0bp \@plus 2bp \@minus 2bp,aboveskip=6bp \@plus 2bp \@minus 2bp}%
\captionsetup[figure]{position=bottom,belowskip=-3bp \@plus 2bp \@minus 2bp,aboveskip=6bp \@plus 2bp \@minus 2bp}%
\captionsetup[subfloat]
{labelformat=simple,font=thu,captionskip=6bp,nearskip=6bp,farskip=0bp,topadjust=0bp}
\renewcommand{\thesubfigure}{(\alph{subfigure})}
\renewcommand{\thesubtable}{(\alph{subtable})}
\let\thu@LT@array\LT@array
\def\LT@array{\thu@LT@array}
%</cls>
%    \end{macrocode}
%
%\subsection{自定环境}
%
% 在这里我们自定义一些论文种会使用到的环境,主要有摘要,符号表,致谢,个人介绍等:
% 这些单独定义的环境可以分别配置以满足要求。
%
% 有些论文需要在正文前面加入符号列表, 其内容格式是简单的列表环境:
% \changes{v2.2}{2011/09/10}{略微减小列表间距,与行距相等}
% \changes{v2.2}{2011/09/13}{略微减小标签和说明间距}
% \changes{v2.3}{2013/12/27}{重新制作了denotation, 参见样例文件}
% \changes{v2.5}{2015/05/11}{增加了denotation中条目的宽度}
% \changes{v2.5}{2015/05/11}{减小了denotation章节标题和条目的间距-4bp}
% \changes{v2.5}{2015/05/11}{修改denotation样式}
%
%    \begin{macrocode}
%<*cls>
\newenvironment{denotation}[1][2.71828cm]{
    \noindent\vskip-4bp\begin{list}{}%
    {\vskip-30bp\xiaosi[1.5]
    \renewcommand\makelabel[1]{\textbf{##1}\hfil}
    \setlength{\labelwidth}{#1} % 标签盒子宽度
    \setlength{\labelsep}{0cm} % 标签与列表文本距离
    \setlength{\itemindent}{0em} % 标签缩进量
    \setlength{\leftmargin}{\labelwidth+\labelsep+2em} % 左边界
    \setlength{\rightmargin}{0cm}
    \setlength{\parsep}{0cm} % 段落间距
    \setlength{\itemsep}{0cm} % 标签间距
    \setlength{\listparindent}{0pt} % 段落缩进量
    \setlength{\topsep}{0pt} % 标签与上文的间距
}}{\end{list}}
%</cls>
%    \end{macrocode}
%
% 致谢往往在正文的最后:
%
% \changes{v1.8}{2010/10/15}{致谢之间要加一个空格}
%    \begin{macrocode}
%<*cls>
\newenvironment{ack}{%
    \chapter*{致\hspace{1em}谢}%
    \addcontentsline{toc}{chapter}{致谢}%
    \ifisanon\color{white}\else\relax\fi%
    \xiaosi%
    \@afterheading}
    {\par\vspace{2em}\par}
%</cls>
%    \end{macrocode}
%
% 个人简历这一部分用来放置作者在研究生期间取得的成果,发表的论文等。可以
% 详细的参考\verb|data/|中的文件自己书写。
% \changes{v1.4}{2009/10/26}{修改标题}
%
%    \begin{macrocode}
%<*cls>
\newenvironment{resume}{%
    \chapter*{作者在学期间取得的学术成果}
    \addcontentsline{toc}{chapter}{作者在学期间取得的学术成果}
    \xiaosi
    \@afterheading}
    {\par\vspace{2em}\par}
%</cls>
%    \end{macrocode}
%
%\subsubsection{定理环境}
% 定理环境可能数学论文中应用较多:
% \changes{v2.0}{2010/11/09}{修改定理的分隔符和QED符号,修改字体;缩进为段落缩进;修改编号}
% \changes{v2.2}{2011/10/14}{设定定理、定义环境的合理间隔}
% \changes{v2.5}{2015/05/11}{去除证明环境的黑色方块QED}
%
%    \begin{macrocode}
%<*cls>
\renewtheoremstyle{nonumberplain}%
{\item[\hspace*{2em} \theorem@headerfont ##1\ \theorem@separator]}%
{\item[\hspace*{2em} \theorem@headerfont ##1\ (##3)\theorem@separator]}
\theoremstyle{nonumberplain}
\theorembodyfont{\kai\xiaosi[1.3]}
\theoremheaderfont{\hei\xiaosi[1.3]}
%\theoremsymbol{\ensuremath{\blacksquare}}
\theoremseparator{:\,}
\newtheorem{proof}{证明}[chapter]
\newtheorem{assumption}{假设}[chapter]

\renewtheoremstyle{plain}%
{\item[\hspace*{2em} \theorem@headerfont ##1\ ##2\theorem@separator]}%
{\item[\hspace*{2em} \theorem@headerfont ##1\ ##2\ (##3)\theorem@separator]}
\theoremstyle{plain}
\theorembodyfont{\kai\xiaosi[1.3]}
\theoremheaderfont{\hei\xiaosi[1.3]}
\theoremsymbol{}
\newtheorem{definition}{定义}[chapter]
\newtheorem{lemma}{引理}[chapter]
\newtheorem{theorem}{定理}[chapter]
\newtheorem{axiom}{公理}[chapter]
\newtheorem{corollary}{推论}[chapter]
\newtheorem{conjecture}{猜想}[chapter]
\newtheorem{proposition}{命题}[chapter]
\newtheorem{exercise}{练习}[section]
\newtheorem{example}{例}[chapter]
\newtheorem{problem}{问题}[section]
\newtheorem{remark}{注释}[section]
%</cls>
%    \end{macrocode}
%
% \changes{v2.5}{2015/05/11}{将example的编号改为chapter}
%
%\subsection{论文属性}
% 这里的内容主要用来定义封面中的一些元素,你可以像填空一样完成封面的制作:
% \changes{v0.993}{2009/08/26}{添加cosupervisor,协助指导导师}
% \changes{v1.3}{2009/10/03}{添加英文第二导师,增加判断语句,在文档开始时执行}
%
%    \begin{macrocode}
%<*cls>
\def\classification#1{\def\@classification{#1}} % 中图分类号
\def\serialno#1{\def\@serialno{#1}} % 学号
\def\UDC#1{\def\@UDC{#1}} % UDC号
\def\confidentiality#1{\def\@confidentiality{#1}} % 密级
\def\title#1{\def\@title{#1}} % 中文题目
\newtoks\displaytitle
\def\author#1{\def\@author{#1}}
\def\zhdate#1{\def\@zhdate{#1}}	% 中文日期
\def\subject#1{\def\@subject{#1}} % 中文学科
\def\researchfield#1{\def\@researchfield{#1}} % 中文研究方向
\def\supervisor#1{\def\@supervisor{#1}} % 导师
\def\cosupervisor#1{\def\@cosupervisor{#1}} % 协助指导教师
\def\papertype#1{\def\@papertype{#1}} % 工学,理学,同等学历申请工(理)学
\def\entitle#1{\def\@entitle{#1}}
\def\enauthor#1{\def\@enauthor{#1}}
\def\ensupervisor#1{\def\@ensupervisor{#1}}
\def\encosupervisor#1{\def\@encosupervisor{#1}}
\def\endate#1{\def\@endate{#1}}
\def\ensubject#1{\def\@ensubject{#1}}
\def\enpapertype#1{\def\@enpapertype{#1}} % Engineering, Science
\def\optionpaperclass#1{\def\@optionpaperclass{#1}} % paperclass
\def\optionpaperclassen#1{\def\@optionpaperclassen{#1}}
\def\optionas#1{\def\@optionas{#1}} % Supervisor
%</cls>
%    \end{macrocode}
%
% 我们看用户是想用博士封面还是硕士封面:
%
%    \begin{macrocode}
%<*cls>
\ifismaster
  \optionpaperclass{硕士}
  \optionpaperclassen{Master}
  \optionas{Supervisor}
\else
  \optionpaperclass{博士}
  \optionpaperclassen{PhD}
  \optionas{Supervisor}
\fi
%</cls>
%    \end{macrocode}
%
%\subsection{制作封面}
%
% 由于封面中一些元素是可选的,如果在正文中没有定义,那么判断ifx的时候就会出错,
% 我们加入下面的命令进行判断,如果没定义,我们就令他为空。
% 这个命令将在文档开始时自动执行。
%
%    \begin{macrocode}
%<*cls>
\newcommand{\checkparameter}
{
  \ifthenelse{\isundefined{\@cosupervisor}}{\cosupervisor{}}{}
  \ifthenelse{\isundefined{\@encosupervisor}}{\encosupervisor{}}{}
}
%</cls>
%    \end{macrocode}
%
% 制作封面比较复杂,需要一些手动调整的东西,首先来看第一页,
% 重新定义了\verb|maketitle|,
% 用表格来安排页面元素,页头采用仿宋五号字体,段前段后间距一行,这个空一行就用3ex实现,
% \changes{v1.5}{2009/11/18}{微调封面布局}
% \changes{v2.0}{2010/11/09}{标题用粗宋}
% \changes{v2.0}{2010/11/09}{增加盲评控制}
% \changes{v2.1}{2010/11/23}{设定页码为alph,增加cleardoublepage,这样在博士双面时方便打印}
% \changes{v2.5}{2015/05/11}{盲评时去除serialno}
%    \begin{macrocode}
%<*cls>
\def\maketitle{%
  \def\entry##1##2##3{%
    \multicolumn{##1}{l}{\underline{\hbox to ##2{\hfil##3\hfil}}}
    }
  \null
  \ifisanon%
  \author{}%
  \serialno{}%
  \enauthor{}%
  \supervisor{}%
  \cosupervisor{}%
  \ensupervisor{}%
  \encosupervisor{}%
  \else\relax\fi%
  \pagenumbering{alph}% not display, for print only
  \thispagestyle{empty}%
  \begin{center}\leavevmode	% 表格环境
  {\fangsong \wuhao[1.25]%
    \begin{tabular}{llcll}
    分类号 	& \entry{1}{3.2cm}{\@classification} & \hspace*{4.8cm}%
    学号   	& \entry{1}{3.2cm}{\@serialno}         \\[5mm]    %
    U\ D\ C	& \entry{1}{3.2cm}{\@UDC} &            \hspace*{4.8cm}
    密级	& \entry{1}{3.2cm}{\@confidentiality}
    \end{tabular}
  }
  \par
  \vspace*{2.5cm} %插入空白
  {\heiti\sanhao \@papertype{}\@optionpaperclass{}学位论文}\\
  \vspace{12bp}
  {\cusong\erhao[1.25] \@title \par}%
  \vspace{45bp} %从WORD中得来
  {\heiti \sihao
    \begin{tabular}{cp{8cm}c}
      \ifismaster
      \raisebox{-3.7ex}[0pt]{\makebox[3cm][s]{硕\hfill{}士\hfill{}生\hfill{}姓\hfill{}名}} &
      \else
      \raisebox{-3.7ex}[0pt]{\makebox[3cm][s]{博\hfill{}士\hfill{}生\hfill{}姓\hfill{}名}} &
      \fi
        {\fs \hfil\raisebox{-3.7ex}[0pt]{\@author}\hfil{}} & \\[3.2ex]
        \cline{2-2}
        \raisebox{-3.7ex}[0pt]{\makebox[3cm][s]{学\hfill{}科\hfill{}专\hfill{}业}} &
        {\fs \hfil\raisebox{-3.7ex}[0pt]{\@subject}\hfil{}} & \\[3.2ex]
        \cline{2-2}
        \raisebox{-3.7ex}[0pt]{\makebox[3cm][s]{研\hfill{}究\hfill{}方\hfill{}向}} &
        {\fs \hfil\raisebox{-3.7ex}[0pt]{\@researchfield}\hfil{}} & \\[3.2ex]
        \cline{2-2}
        \raisebox{-3.7ex}[0pt]{\makebox[3cm][s]{指\hfill{}导\hfill{}教\hfill{}师}} &
        {\fs \hfil\raisebox{-3.7ex}[0pt]{\@supervisor}\hfil{}} & \\[3.2ex]
        \cline{2-2}
      \ifx\@cosupervisor\@empty\else
      \raisebox{-3.7ex}[0pt]{\makebox[3cm][s]{协\hfill{}助\hfill{}指\hfill{}导\hfill{}教\hfill{}师}} &
          {\fs \hfil\raisebox{-3.7ex}[0pt]{\@cosupervisor}\hfil{}} & \\[3.2ex]
          \cline{2-2}
      \fi
    \end{tabular}
  }
  \end{center}%

  \par
  \vfill
  {\centering \cusong \sanhao \ifisanon{~}\else{国防科技大学研究生院}\fi\\[0.8em]
    {\@zhdate \par}%
  }
  \vspace{1mm}
%</cls>
%    \end{macrocode}
%
%第二页主要是论文的英文信息,简称英文封面
%
%    \begin{macrocode}
%<*cls>
  \cleardoublepage%
  \newpage
  \thispagestyle{empty}%

  \begin{center}\leavevmode
  \vfill\bfseries
  {\erhao[1.25] \@entitle \par}
  {\sanhao[1.25]
  \vfill\vfill\vfill\vfill\vfill\vfill
  \begin{tabular}{rl}
    Candidate:\ & {\textsf{\@enauthor}}\\
    \@optionas{}:\ & {\textsf{\@ensupervisor}}\\
    \ifx\@encosupervisor\@empty\else
      Associate \@optionas{}:\ & {\textsf{\@encosupervisor}} \\
    \fi
  \end{tabular}}
  \vfill\vfill\vfill\vfill
  {\sanhao[1.5]
A dissertation\\
Submitted in partial fulfillment of the requirements\\
for the degree of \textsf{\@optionpaperclassen{} of \@enpapertype}\\
in \textsf{\@ensubject}\\
\makebox[\textwidth]{\ifisanon{~}\else{Graduate School of National University of %
Defense Technology}\fi}\\
\ifisanon{~}\else{Changsha, Hunan, P.\ R.\ China}\fi\\[5mm]
~\@endate~
  }
  \end{center}\vfill
  \cleardoublepage%
%</cls>
%    \end{macrocode}
%
% 第三页放置独创性声明,这里要使用\verb|displaytitle|这个论文元素:
% \changes{v2.1}{2010/12/29}{表格字体定为五号}
%    \begin{macrocode}
%<*cls>
  \newpage
  \thispagestyle{empty}

  {\cusong \erhao \centering \ziju{12pt} 独创性声明 \par\vspace{2cm}}
    \renewcommand{\baselinestretch}{1.5}%
  {\fangsong\xiaosi %
本人声明所呈交的学位论文是我本人在导师指导下进行的研究工
作及取得的研究成果。尽我所知,除文中特别加以标注和致谢的地方外,论文中
不包含其他人已经发表和撰写过的研究成果,也不包含为获得国防科技大学或
其他教育机构的学位或证书而使用过的材料。与我一同工作的同志对本研究所做的
任何贡献均已在论文中作了明确的说明并表示谢意。\par
学位论文题目:\vbox{\hbox to11cm{\hfil \the\displaytitle \hfil}
  \protect\vspace{0.6truemm}\relax
  \hrule depth0pt height0.15truemm width11cm}\par
学位论文作者签名:\hrulefill\hrulefill\hrulefill\hrulefill\hrulefill
  \hfill 日期:\hfill\hfill 年\hfill 月 \hfill 日\hspace{1cm}\par}

  \vspace*{2cm}
  {\cusong \erhao \centering 学位论文版权使用授权书\par\vspace{2cm}}
  {\fangsong\xiaosi %
本人完全了解国防科技大学有关保留、使用学位论文的规定。
本人授权国防科技大学可以保留并向国家有关部门或机构送交论文的复印件和电子
文档,允许论文被查阅和借阅; 可以将学位论文的全部或部分内容编入有关数据库
进行检索,可以采用影印、缩印或扫描等复制手段保存、汇编学位论文。\par
(保密学位论文在解密后适用本授权书。)\par
学位论文题目:\vbox{\hbox to11cm{\hfil \the\displaytitle \hfil}
  \protect\vspace{0.6truemm}\relax
  \hrule depth0pt height0.15truemm width11cm}\par
学位论文作者签名:\hrulefill\hrulefill\hrulefill\hrulefill\hrulefill
  \hfill 日期:\hfill\hfill 年\hfill 月 \hfill 日\par
作者指导教师签名:\hrulefill\hrulefill\hrulefill\hrulefill\hrulefill
  \hfill 日期:\hfill\hfill 年\hfill 月 \hfill 日\par}

  \normalsize % normal, 正文开始
  \def\@tabular{\wuhao[1.25]\old@tabular} % 之后表格字体使用5号
  \cleardoublepage%
  \newpage
  \thispagestyle{empty}

} % QED
%</cls>
%    \end{macrocode}
%
% \Finale
%
% \iffalse meta-comment
% 下面的配置为本文档使用的模板
% \fi
%
% \iffalse
% \changes{v1.8}{2010/10/15}{按照milksea的说法,修改xltxtra和xunicode相对于xeCJK的顺序}
% \changes{v1.8}{2010/10/15}{添加了xeCJKsetcharclass,使itemize环境正确}
%<*nudtx>
%    \begin{macrocode}
%    \end{macrocode}
%</nudtx>
% \fi
%
\endinput
